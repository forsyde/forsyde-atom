\documentclass{standalone}
\usepackage{forsyde-pc}
\usepackage{forsyde-math}
\begin{document}

\newsavebox\formula
\begin{lrbox}{\formula}
\begin{tikzpicture}[nomoccolor, type style = \scriptsize\texttt]
  \primitive[f=pc]{p1}{0,1} {$\fMapS$};
  \primitive[]    {p2}{0,0} {$\scanS$};
  \primitive[]    {p3}{0,-2}{$\pipeS$};
  \vector[] (p1.center) -> (p2);
  \vector[token=(scalar), token pos=.6] (p2.center) - (0,-1);
  \vector[] (p3.center) -> (1.5,-2);
  \primitive[f=pc]{p1}{0,1} {$\fMapS$};
  \primitive[]    {p2}{0,0} {$\scanS$};
  \primitive[]    {p3}{0,-2}{$\pipeS$};

  \node[anchor=east] (a) at (-1.5,.5) {$ \lambda\ s_1 $};
  \node[anchor=north east] () at (1,2) {$ \fmap $};
  \node[] (f)  at (-3,1.5) {$ \fopt{\vect{\vect{f}^n}} $};        
  \node[] (s)  at (-3,1)   {$ \fopt{\vect{\vect{s_x}^n}} $};  
  \node[] (s2) at (-3,0)   {$ \vect{s}_2 $};
  \node[] (s1) at (-3,-2)  {$ \vect{s}_1 $};

  \vectorn[] (f) - (-1.5,1.5);
  \vectorn[] (s) - (-1.5,1);
  \vector[token=scalar] (s2) - (-1.5,0);
  \vectorn[] (-1.5,1.5) -> (p1);
  \vectorn[token=scalar] (-1.5,1) -> (p1);
  \vector[token=scalar] (a) -> (p1);
  \signal[token=scalar] (-1.5,0) -> (p2);
  \draw[dashed] (-1.5,2) rectangle (1,-1);
  \vector[token=(vector)] (0,-1) -> (p3);        
  \vector[] (s1) -> (p3);
\end{tikzpicture}
\end{lrbox}


\newsavebox\network
\begin{lrbox}{\network}
\begin{tikzpicture}[nomoccolor, type style = \scriptsize\texttt]
   \pgfmathsetmacro\c{1.5}
   \pgfmathsetmacro\d{2}
   \pgfmathsetmacro{\w}{\d * 2 - \c};
   \foreach \i in {1,2}{
     \foreach \j in {1,2}{
       \pgfmathsetmacro{\y}{-\i * \c};
       \pgfmathsetmacro{\x}{\j * 2 * \d + \c * \i };
       \ifthenelse{\j=2}{
         \leafstd[ni=3, no=1, nf=1, f1=$ f_{\i n}^n $, type=pc]{p\i\j}{\x,\y}{$p_{\i n}$};
         \node[] (s\i\j) at ($(p\i\j -p.i1)-(1,0)$) {$s_{x,\i n}$};
       }{
         \leafstd[ni=3, no=1, nf=1, f1=$ f_{\i\j}^n $, type=pc]{p\i\j}{\x,\y}{$p_{\i\j}$};
         \node[] (s\i\j) at ($(p\i\j -p.i1)-(1,0)$) {$s_{x,\i\j}$};
       }
       \signaln[] (s\i\j) -> (p\i\j -p.i1);
     }
     \node[] at ($(p\i 1-p.o1)!.4!(p\i 2-p.o1)$) {$\cdots$};
   } 
   \coordinate (a) at ($(p11-p.i1)+(-.4,1)$); \coordinate (b) at ($(p12-p.i1)+(-.4,1)$);
   \signal[|-] (a) -> (p11-p.i2); \signal[-|-] (p11-p.o1) -> (p21-p.i2); 
   \signal[|-] (b) -> (p12-p.i2); \signal[-|-] (p12-p.o1) -> (p22-p.i2); 
   \signal[-|-] (p21-p.o1) -> ($(p21-p.i2) + (\c,-\c)$); 
   \signal[-|-] (p22-p.o1) -> ($(p22-p.i2) + (\c,-\c)$);
   \signal[-|-] (p11-p.o1) -> ($(p12-p.i3) - (\w,0)$); 
   \signal[-|-] (p21-p.o1) -> ($(p22-p.i3) - (\w,0)$); 
   \signal[-|-=.2] ($(p11-p.o1)+(\w,0)$) -> (p12-p.i3); 
   \signal[-|-=.2] ($(p21-p.o1)+(\w,0)$) -> (p22-p.i3); 

   \coordinate (c) at ($(p21-p.o1)+(\d,-\w)$); \coordinate (d) at ($(p22-p.o1)+(\d,-\w)$);
   \vector[shorten <= -1pt] (c) -> ($(d)+(.5,0)$);
   \signal[shorten >= 2pt] ($(c)+(0,.5)$) -> (c); 
   \signal[shorten >= 2pt] ($(d)+(0,.5)$) -> (d);
   \node[] at ($(c)!.35!(d)+(0,.9)$) {$\ddots$};
   \node[] at ($(c)+(-.5,.9)$) {$\ddots$};
   \node[] at ($(d)+(-.5,.9)$) {$\ddots$};

   \node[] (v1) at ($(s11)+(-1,1)$) {$\vect{s}_1$};
   \vector[shorten >= -1pt] (v1) -> (b);

   \coordinate (e) at ($(p11-p.i3)-(1,0)$);
   \coordinate (f) at ($(p21-p.i3)-(\w,0)$);
   \coordinate (g) at ($(f)-(0,\d)$);

   \signal[] (e) -> (p11-p.i3);
   \signal[] (f) -> (p21-p.i3);
   \vector[shorten <= -1pt] (e) - (g);
   \node[] (v2) at ($(e)!.5!(g)-(1,0)$) {$\vect{s}_2$};
   \vector[] (v2) -> ($(e)!.5!(g)$);
\end{tikzpicture}
\end{lrbox}

\begin{tikzpicture}[nomoccolor, type style = \scriptsize\texttt]
  \node[anchor=east] (f) at (0,0) {\usebox{\formula}};
  \node[anchor=west] (n) at (2,0) {\usebox{\network}};
  \node[anchor=west]     at (1,0) {$\equiv$};
\end{tikzpicture}
\end{document}

%%% Local Variables:
%%% mode: latex
%%% TeX-master: t
%%% End:
