\documentclass{standalone}
\usepackage{forsyde-tikz}
\usepackage{forsyde-plot}


\def\sigsini(#1){sin(10 * deg(#1))+1}
\def\sigsinii(#1){sin(10 * deg(#1))+2}
\def\sigsiniii(#1){sin(10 * deg(#1))+3}
\def\sigsin(#1){sin(10 * deg(#1))}
\begin{document}

\newsavebox\outsig
\begin{lrbox}{\outsig}
  \begin{ct-signal}[xmin=0, xmax=1, ymin=-.5, ymax=4]{2}
    \ctsignal{$S_{o,1}$}\ctf{0}{.3}{\sigsin}\ctf{.3}{.6}{\sigsini}
                       \ctf{.6}{.9}{\sigsinii}\ctf{.9}{1}{\sigsiniii}\ctend  
  \end{ct-signal} 
\end{lrbox}


\begin{tikzpicture}[]
  \leafstd[moc=ct, ni=1, no=1, nf=2, f1={$(+1)$}, f2={$(.3, sin)$},
           type=generate]{c1} {0,0} {}; 
  \node[anchor=west, scale=.7] (o1) at ($(c1)+(3,0)$) {\usebox{\outsig}};
  \node[anchor=west] (o1l) at ($(c1-p.o1)+(1,0)$) {$S_{o,1}$};
  \signal[] (c1-p.o1) -> (o1l);
\end{tikzpicture}  
\end{document}

%%% Local Variables:
%%% mode: latex
%%% TeX-master: t
%%% End:
