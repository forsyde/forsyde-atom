\documentclass{standalone}
\usepackage{forsyde-pc}
\usepackage{forsyde-math}
\begin{document}

\newsavebox\formula
\begin{lrbox}{\formula}
\begin{tikzpicture}[nomoccolor, type style = \scriptsize\texttt]
  \patternnodestd[ni=1, no=1, outer shape=transition shape v1gv1, type=inits]{init}{-2,0};
\end{tikzpicture}
\end{lrbox}

\newsavebox\network
\begin{lrbox}{\network}
\begin{tikzpicture}[nomoccolor, type style = \scriptsize\texttt, global scale = .7]
\vector[] (0,.5) - (0,4.5); \vector[] (2.2,0.5) - (2.2,4.5);
\foreach \i in {1,2,3,4} {
  \vector[shorten >= 1pt, shorten <= 1pt] ($(2,\i)-(0,.5)$) - ($(2,\i)+(0,.5)$);
  \signal[] (-1,\i) -> (0,\i);
  \vector[] (2,\i) -> (3,\i);
  \foreach \j in {1,...,\i} {
    \pgfmathsetmacro{\y}{\j + (\i - \j - (4 - \i) - (4 - \j) + (4 - \j) ) / 16}
    \signal[] (0,\i) -> (2,\y);
  }
}
\end{tikzpicture}
\end{lrbox}

\begin{tikzpicture}[nomoccolor, type style = \scriptsize\texttt]
  \node[anchor=east] (f) at (0,0) {\usebox{\formula}};
  \node[anchor=west] (n) at (2,0) {\usebox{\network}};
  \node[anchor=west]     at (1,0) {$\equiv$};
\end{tikzpicture}
\end{document}

%%% Local Variables:
%%% mode: latex
%%% TeX-master: t
%%% End:
