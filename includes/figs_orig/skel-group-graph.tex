\documentclass{standalone}
\usepackage{forsyde-pc}
\usepackage{forsyde-math}
\begin{document}

\newsavebox\formula
\begin{lrbox}{\formula}
\begin{tikzpicture}[nomoccolor, type style = \scriptsize\texttt]
  \patternnodestd[ni=1, no=1, nf=1, f1={$2$}, outer shape=transition shape v1gv1, type=group]{p}{-2,0};
\end{tikzpicture}
\end{lrbox}

\newsavebox\network
\begin{lrbox}{\network}
\begin{tikzpicture}[nomoccolor, type style = \scriptsize\texttt, global scale = .5]
  \vector[] (0,0.9) - (0,6.1); \vector[] (2.2,0.9) - (2.2,6.1);
  \foreach \i/\j in {1/2,3/4,5/6} {
    \vector[shorten >= -2pt, shorten <= -2pt] ($(2,\i)$) - ($(2,\j)$);
    \signal[] (-1,\i) -> (0,\i);
    \signal[] (-1,\j) -> (0,\j);
    \vector[] ($(2,\i)+(0,.5)$) -> ($(4,\i)+(0,.5)$);
    \signal[] (0,\i) -> (2,\i);
    \signal[] (0,\j) -> (2,\j);
  }
\end{tikzpicture}
\end{lrbox}

\begin{tikzpicture}[nomoccolor, type style = \scriptsize\texttt]
  \node[anchor=east] (f) at (0,0) {\usebox{\formula}};
  \node[anchor=west] (n) at (2,0) {\usebox{\network}};
  \node[anchor=west]     at (1,0) {$\equiv$};
\end{tikzpicture}
\end{document}

%%% Local Variables:
%%% mode: latex
%%% TeX-master: t
%%% End:
