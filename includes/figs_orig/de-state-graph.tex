\documentclass{standalone}
\usepackage{forsyde-plot}
\usepackage{forsyde-tikz}

\begin{document}
\newsavebox\insig
\sbox{\insig}{
  \begin{de-signal}[%
    timestamps=5,tag scale=.1]{24}
    \tiny
    \nextsignal{$S_{i}$} \present{9}{2} \present{15}{3} \infinity 
  \end{de-signal}
}
\newsavebox\outsig
\sbox{\outsig}{
  \begin{de-signal}[%
    timestamps=5, tag scale=.3]{24}
    \tiny
    \nextsignal{$S_{o}$}  \present{5}{4} \present{4}{6} 
        \present{1}{7} \present{4}{9} \present{1}{10} 
        \present{3}{12} \present{1}{13} \present{1}{14} 
        \present{3}{16} \present{1}{17} 
        \infinity 
  \end{de-signal}
}

% { 2 @0, 3 @9}
% { 4 @0, 6 @5, 7 @9, 9 @10, 10 @14, 12 @15, 13 @18, 14 @19, 16 @20, 17 @23}


\begin{tikzpicture}[]
  \leafstd[moc=de, ni=1, no=1, nf=2, f1={$(+)$}, f2={$(5,2)$}, type=state]{c1} {0,0} {};
  \node[anchor=east] (i) at ($(c1.west)-(1,0)$) {\usebox{\insig}};
  \node[anchor=west] (o) at ($(c1.east)+(1,0)$) {\usebox{\outsig}};

  \signal[-|-=.8] (i) -> (c1-p.i1);
  \signal[] (c1-p.o1) -> (o);
\end{tikzpicture}
\end{document}

%%% Local Variables:
%%% mode: latex
%%% TeX-master: t
%%% End:
