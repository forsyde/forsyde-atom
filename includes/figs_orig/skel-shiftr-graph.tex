\documentclass{standalone}
\usepackage{forsyde-pc}
\usepackage{forsyde-math}
\begin{document}

\newsavebox\formula
\begin{lrbox}{\formula}
\begin{tikzpicture}[nomoccolor, type style = \scriptsize\texttt]
  \patternnodestd[ni=1, no=1, outer shape=transition shape v1v1, type=shiftr]{p}{-2,0};
\end{tikzpicture}
\end{lrbox}

\newsavebox\network
\begin{lrbox}{\network}
\begin{tikzpicture}[nomoccolor, type style = \scriptsize\texttt, global scale = .5]
  \vector[] (0,1) - (0,4); 
  \vector[] (2,1) - (2,4);
  \foreach \i in {1,2,3,4} {
    \signal[] (-1,\i) -> (0,\i);
    \signal[] (2,\i) -> (3,\i);
    \signal[] ($(0,\i)+(0,1)$) -> (2,\i);
  }
  \signal[] (-1,5) - (0,5);
\end{tikzpicture}
\end{lrbox}

\begin{tikzpicture}[nomoccolor, type style = \scriptsize\texttt]
  \node[anchor=east] (f) at (0,0) {\usebox{\formula}};
  \node[anchor=west] (n) at (2,0) {\usebox{\network}};
  \node[anchor=west]     at (1,0) {$\equiv$};
\end{tikzpicture}
\end{document}

%%% Local Variables:
%%% mode: latex
%%% TeX-master: t
%%% End:
