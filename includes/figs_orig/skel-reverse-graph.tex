\documentclass{standalone}
\usepackage{forsyde-pc}
\usepackage{forsyde-math}
\begin{document}

\newsavebox\formula
\begin{lrbox}{\formula}
\begin{tikzpicture}[nomoccolor, type style = \scriptsize\texttt]
  \patternnodestd[type=reverse, ni=1, no=1, outer shape=transition shape v1v1]{p}{0,0};
\end{tikzpicture}
\end{lrbox}

\newsavebox\network
\begin{lrbox}{\network}
\begin{tikzpicture}[nomoccolor, type style = \scriptsize\texttt]
  \patternnodecustom[ni=6, no=6, outer shape=transition shape v1v1]{c1}{1,0};
  \patternnodecustom[ni=6, no=6, outer shape=transition shape v1v1]{c2}{2,0};
  \foreach \i/\j in {1,...,6} {
   \pgfmathtruncatemacro\j{7-\i}
    \signal[] ($(c1-p.i\i)-(1,0)$) -> (c1-p.i\i);
    \signal[] (c2-p.o\i) -> ($(c2-p.o\i)+(1,0)$);
    \signal[] (c1-p.o\i) -> (c2-p.i\j);
  }
\end{tikzpicture}
\end{lrbox}

\begin{tikzpicture}[nomoccolor, type style = \scriptsize\texttt]
  \node[anchor=east] (f) at (0,0) {\usebox{\formula}};
  \node[anchor=west] (n) at (2,0) {\usebox{\network}};
  \node[anchor=west]     at (1,0) {$\equiv$};
\end{tikzpicture}
\end{document}
%%% Local Variables:
%%% mode: latex
%%% TeX-master: t
%%% End:
