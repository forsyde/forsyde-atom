\documentclass{standalone}
\usepackage{forsyde-pc}
\usepackage{forsyde-math}
\begin{document}

\newsavebox\formula
\begin{lrbox}{\formula}
\begin{tikzpicture}[nomoccolor, type style = \scriptsize\texttt]
  \patternnodestd[ni=2, no=1, nf=1, f1={$\vect{2,3,6}$}, outer shape=transition shape v2v1, type=scatter]{p}{-2,0};
\end{tikzpicture}
\end{lrbox}

\newsavebox\network
\begin{lrbox}{\network}
\begin{tikzpicture}[nomoccolor, type style = \scriptsize\texttt, global scale = .5]
  \vector[] (0,.5) - (0,3); \vector[] (2,.5) - (2,3);
  \vector[] (0,0) - (0,-2);
  \foreach \i in {1,...,6} {
    \pgfmathsetmacro{\y}{\i * .5}
    \signal[] (-1,\y) -> (0,\y);
    \signal[] (2,\y) -> (3,\y);
  }
  \foreach \i [count=\n from 0] in {5,4,1} {
    \pgfmathsetmacro{\y}{\i * .5}
    \signal[] (-1,-\n) -> (0,-\n);
    \signal[] (0,-\n) -> (2,\y);
  } 
  \foreach \i in {2,3,6} {
    \pgfmathsetmacro{\y}{\i * .5}
    \signal[] (0,\y) -> (2,\y);
  } 
\end{tikzpicture}
\end{lrbox}

\begin{tikzpicture}[nomoccolor, type style = \scriptsize\texttt]
  \node[anchor=east] (f) at (0,0) {\usebox{\formula}};
  \node[anchor=west] (n) at (2,0) {\usebox{\network}};
  \node[anchor=west]     at (1,0) {$\equiv$};
\end{tikzpicture}
\end{document}

%%% Local Variables:
%%% mode: latex
%%% TeX-master: t
%%% End:

