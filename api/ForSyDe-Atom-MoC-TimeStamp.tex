\haddockmoduleheading{ForSyDe.Atom.MoC.TimeStamp}
\label{module:ForSyDe.Atom.MoC.TimeStamp}
\haddockbeginheader
{\haddockverb\begin{verbatim}
module ForSyDe.Atom.MoC.TimeStamp (
    TimeStamp,  picosec,  nanosec,  microsec,  milisec,  sec,  minutes,  hours, 
    toTime,  pi, 
  ) where\end{verbatim}}
\haddockendheader

This module implements a timestamp data type, based on
 \haddocktt{Data.Time.Clock}. \par

\begin{haddockdesc}
\item[\begin{tabular}{@{}l}
type\ TimeStamp\ =\ DiffTime
\end{tabular}]\haddockbegindoc
Alias for the type representing discrete time. It is inherently
 quantizable, the quantum being a picosecond (10⁻¹²
 seconds), thus it can be considered order-isomorphic with a set of
 integers, i.e. between any two timestamps there is a finite number
 of timestamps. Moreover, a timestamp can be easily translated into
 a rational number representing fractions of a second, so the
 conversion between timestamps (discrete time) and rationals
 (analog/continuous time) is straightforward.\par
This type is used in the explicit tags of the
 \haddockid{DE} MoC (and subsequently the discrete event
 evaluation engine for simulating the \haddockid{CT} MoC).\par

\end{haddockdesc}
\begin{haddockdesc}
\item[\begin{tabular}{@{}l}
picosec\ ::\ Integer\ ->\ TimeStamp
\end{tabular}]\haddockbegindoc
Specifies a timestamp in terms of picoseconds.\par

\end{haddockdesc}
\begin{haddockdesc}
\item[\begin{tabular}{@{}l}
nanosec\ ::\ Integer\ ->\ TimeStamp
\end{tabular}]\haddockbegindoc
Specifies a timestamp in terms of nanoseconds.\par

\end{haddockdesc}
\begin{haddockdesc}
\item[\begin{tabular}{@{}l}
microsec\ ::\ Integer\ ->\ TimeStamp
\end{tabular}]\haddockbegindoc
Specifies a timestamp in terms of microseconds.\par

\end{haddockdesc}
\begin{haddockdesc}
\item[\begin{tabular}{@{}l}
milisec\ ::\ Integer\ ->\ TimeStamp
\end{tabular}]\haddockbegindoc
Specifies a timestamp in terms of miliseconds.\par

\end{haddockdesc}
\begin{haddockdesc}
\item[\begin{tabular}{@{}l}
sec\ ::\ Integer\ ->\ TimeStamp
\end{tabular}]\haddockbegindoc
Specifies a timestamp in terms of seconds.\par

\end{haddockdesc}
\begin{haddockdesc}
\item[\begin{tabular}{@{}l}
minutes\ ::\ Integer\ ->\ TimeStamp
\end{tabular}]\haddockbegindoc
Specifies a timestamp in terms of minutes.\par

\end{haddockdesc}
\begin{haddockdesc}
\item[\begin{tabular}{@{}l}
hours\ ::\ Integer\ ->\ TimeStamp
\end{tabular}]\haddockbegindoc
Specifies a timestamp in terms of hours.\par

\end{haddockdesc}
\begin{haddockdesc}
\item[\begin{tabular}{@{}l}
toTime\ ::\ TimeStamp\ ->\ Rational
\end{tabular}]\haddockbegindoc
Converts a timestamp to a rational number, used for describing
 continuous time.\par

\end{haddockdesc}
\begin{haddockdesc}
\item[\begin{tabular}{@{}l}
pi\ ::\ TimeStamp
\end{tabular}]\haddockbegindoc
\haddockid{TimeStamp} representation of the number π. Converted from
 the \haddocktt{Prelude} equivalent, which is \haddockid{Floating}.\par

\end{haddockdesc}