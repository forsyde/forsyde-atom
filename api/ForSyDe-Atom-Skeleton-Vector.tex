\haddockmoduleheading{ForSyDe.Atom.Skeleton.Vector}
\label{module:ForSyDe.Atom.Skeleton.Vector}
\haddockbeginheader
{\haddockverb\begin{verbatim}
module ForSyDe.Atom.Skeleton.Vector (
    Vector(Null, (:>)),  null,  unit,  (<++>),  vector,  fromVector,  indexes, 
    isNull,  (<:),  farm22,  reduce,  prefix,  suffix,  pipe,  (=/=),  recur, 
    cascade2,  mesh2,  length,  index,  fanout,  fanoutn,  generate,  iterate, 
    first,  last,  inits,  tails,  init,  tail,  concat,  reverse,  group, 
    shiftr,  shiftl,  rotr,  rotl,  take,  drop,  takeWhile,  filterIdx,  odds, 
    evens,  stride,  get,  (<@),  (<@!),  gather1,  (<@>),  replace,  scatter, 
    bitrev,  duals,  unduals,  zipx,  unzipx
  ) where\end{verbatim}}
\haddockendheader

This module defines the data type \haddockid{Vector} as a categorical type,
 and implements the atoms for the \haddockid{Skeleton}
 class. Algorithmic skeletons for \haddockid{Vector} are mostly described in
 their factorized form, which ensures that they are catamorphisms
 (see the \href{ForSyDe-Atom-Skeleton.html#factorization}{factorization}
 theorem). Where efficiency or practicality is a concern, some
 skeletons are implemented as recurrences. One can still prove that
 they are catamorphisms through alternative theorems (see
 \href{ForSyDe-Atom.html#skillicorn05}{[Skillicorn05]}).\par
\textbf{IMPORTANT!!!}
 see the \href{ForSyDe-Atom.html#naming_conv}{naming convention} rules
 on how to interpret, use and develop your own constructors.\par

\section{Vector data type}
\begin{haddockdesc}
\item[\begin{tabular}{@{}l}
data\ Vector\ a
\end{tabular}]\haddockbegindoc
\haddockbeginconstrs
\haddockdecltt{=} & \haddockdecltt{Null} & Null element. Terminates a vector. \\
\haddockdecltt{|} & \haddockdecltt{a (:>) (Vector a)} & appends an element at the head of a vector. \\
\end{tabulary}\par
\haddockbeginconstrs
\end{tabulary}\par
The  \haddockid{Vector}, or at least its interpretation, is the
 exact equivalent of an infinite list, as defined in
 \href{ForSyDe-Atom.html#bird97}{[Bird97]}. Its name though is borrowed
 from \href{ForSyDe-Atom.html#reekie95}{[Reekie95]}, since it is more
 suggestive in the context of process networks.\par
According to \href{ForSyDe-Atom.html#bird97}{[Bird97]}, \haddockid{Vector}
 should be implemented as following:\par
\begin{quote}
{\haddockverb\begin{verbatim}
data Vector a = Null                   -- null element
              | Unit a                 -- singleton vector
              | Vector a <++> Vector a -- concatenate two vectors\end{verbatim}}
\end{quote}
This construction suggests the possibility of splitting a \haddockid{Vector}
 into multiple parts and evaluating it in parallel. Due to reasons
 of efficiency, and to ensure that the structure is flat and
 homogeneous, \haddockid{Vector} is implemented using the same constructors as
 an infinite list like in \href{ForSyDe-Atom.html#bird87}{[Bird87]} (see
 below). When defining skeletons of vectors we will not use the real
 constructors though, but the theoretical ones defined above and
 provided as \href{#g:2}{functions} . This way we align ForSyDe-Atom's
 \haddockid{Vector} type with the categorical type theory and its theorems.\par
Another particularity of \haddockid{Vector} is that it instantiates the
 reduction atom \haddockid{={\char '134}=} as a \emph{right fold}, as it is the most efficient
 implementation in the context of lazy evaluation. As a consequence
 reduction is performed \textbf{\emph{from right to left}}. This is noticeable
 especially in the case of pipeline-based skeletons (where \haddockid{pipe}
 itself is a reduction with the right-associative composition
 operator \haddockid{.}) is performed from right to left, which
 comes in natural when considering the order of function
 composition. Thus for \haddockid{reduce}-based skeletons (e.g. \haddocktt{prefix},
 \haddocktt{suffix}, \haddocktt{recur}, \haddocktt{cascade}, \haddocktt{mesh}) the result vectors shall be
 read from end to beginning.\par

\end{haddockdesc}
\begin{haddockdesc}
\item[\begin{tabular}{@{}l}
instance\ Functor\ Vector
\end{tabular}]\haddockbegindoc
Provides an implementation for \haddockid{=.=}.\par

\end{haddockdesc}
\begin{haddockdesc}
\item[\begin{tabular}{@{}l}
instance\ Applicative\ Vector
\end{tabular}]\haddockbegindoc
Provides an implementation for \haddockid{=*=}.\par

\end{haddockdesc}
\begin{haddockdesc}
\item[\begin{tabular}{@{}l}
instance\ Foldable\ Vector
\end{tabular}]\haddockbegindoc
Provides an implementation for \haddockid{={\char '134}=}.\par

\end{haddockdesc}
\begin{haddockdesc}
\item[\begin{tabular}{@{}l}
instance\ Skeleton\ Vector
\end{tabular}]\haddockbegindoc
Ensures that \haddockid{Vector} is a structure associated with the Skeleton Layer.\par

\end{haddockdesc}
\begin{haddockdesc}
\item[\begin{tabular}{@{}l}
instance\ Eq\ a\ =>\ Eq\ (Vector\ a)
\end{tabular}]
\end{haddockdesc}
\begin{haddockdesc}
\item[\begin{tabular}{@{}l}
instance\ Read\ a\ =>\ Read\ (Vector\ a)
\end{tabular}]\haddockbegindoc
The vector 1 :> 2 :> Null is read using the string "<1,2>".\par

\end{haddockdesc}
\begin{haddockdesc}
\item[\begin{tabular}{@{}l}
instance\ Show\ a\ =>\ Show\ (Vector\ a)
\end{tabular}]\haddockbegindoc
The vector 1 :> 2 :> Null is represented as <1,2>.\par

\end{haddockdesc}
\begin{haddockdesc}
\item[\begin{tabular}{@{}l}
instance\ Plottable\ a\ =>\ Plottable\ (Vector\ a)
\end{tabular}]\haddockbegindoc
Vectors of plottable types\par

\end{haddockdesc}
\begin{haddockdesc}
\item[\begin{tabular}{@{}l}
instance\ Plottable\ a\ =>\ Plot\ (Vector\ a)
\end{tabular}]\haddockbegindoc
vectors of coordinates\par

\end{haddockdesc}
\section{"Constructors"}
Theoretical constructors for the \haddockid{Vector} type, used in the
 definition of skeletons as catamorphisms.\par

\begin{haddockdesc}
\item[\begin{tabular}{@{}l}
null\ ::\ Vector\ a
\end{tabular}]\haddockbegindoc
Constructs a null vector.\par
\begin{quote}
{\haddockverb\begin{verbatim}
>>> null
<>

\end{verbatim}}
\end{quote}
\end{haddockdesc}
\begin{haddockdesc}
\item[\begin{tabular}{@{}l}
unit\ ::\ a\ ->\ Vector\ a
\end{tabular}]\haddockbegindoc
Constructs a singleton vector.\par
\begin{quote}
{\haddockverb\begin{verbatim}
>>> unit 1
<1>

\end{verbatim}}
\end{quote}
\end{haddockdesc}
\begin{haddockdesc}
\item[\begin{tabular}{@{}l}
(<++>)\ ::\ Vector\ a\ ->\ Vector\ a\ ->\ Vector\ a
\end{tabular}]\haddockbegindoc
Constructs a vector by appending two existing vectors.\par
\begin{quote}
{\haddockverb\begin{verbatim}
>>> unit 1 <++> unit 2
<1,2>

\end{verbatim}}
\end{quote}
\end{haddockdesc}
\section{Utilities}
\begin{haddockdesc}
\item[\begin{tabular}{@{}l}
vector\ ::\ {\char 91}a{\char 93}\ ->\ Vector\ a
\end{tabular}]\haddockbegindoc
Converts a list to a vector.\par

\end{haddockdesc}
\begin{haddockdesc}
\item[\begin{tabular}{@{}l}
fromVector\ ::\ Vector\ a\ ->\ {\char 91}a{\char 93}
\end{tabular}]\haddockbegindoc
Converts a vector to a list.\par

\end{haddockdesc}
\begin{haddockdesc}
\item[\begin{tabular}{@{}l}
indexes\ ::\ Vector\ Integer
\end{tabular}]\haddockbegindoc
Creates the infinite vector:\par
\begin{quote}
{\haddockverb\begin{verbatim}
<1,2,3,4,...>\end{verbatim}}
\end{quote}
Used mainly for operation on indexes.\par

\end{haddockdesc}
\begin{haddockdesc}
\item[\begin{tabular}{@{}l}
isNull\ ::\ Vector\ a\ ->\ Bool
\end{tabular}]\haddockbegindoc
Returns \haddocktt{True} if the argument is a null vector.\par

\end{haddockdesc}
\begin{haddockdesc}
\item[\begin{tabular}{@{}l}
(<:)\ ::\ Vector\ a\ ->\ a\ ->\ Vector\ a
\end{tabular}]\haddockbegindoc
Appends an element at the end of a vector.\par

\end{haddockdesc}
\section{Skeletons}
Algorithmic skeletons on vectors are mainly presented in terms
 of compositions of the atoms associated with the
 \haddocktt{Skeleton} Layer. When defining them,
 we use the following operators:\par
(image: fig/eqs-skel-vector-operators.png)\par
where:\par
\begin{itemize}
\item
(1) is the \haddockid{unit} constructor, constructing a singleton vector.\par

\item
(2) is the \haddockid{<++>} constructor, concatenating two vectors.\par

\item
(3) is the \haddocktt{<@!>} selector. The subscript notation is used to
 denote element at position \emph{n} in a vector.\par

\item
(4) suggests an arbitrary selector which returns a vector with
 another one's elements, based on some indices. The shown example
 is an alternative notation for the \haddockid{tail} skeleton.\par

\end{itemize}

\subsection{Functional networks}
This sub-category denotes skeletons (patterns) which are take
 functions as arguments. If the functions are
 \haddockid{MoC} layer entities, i.e. processes, then these
 patterns are capable of constructing parallel process
 networks. Using the applicative mechanism, the designer has a
 high degree of freedom when customizing process networks through
 systematic partial application, rendering numerous possible
 usages for the same pattern. To avoid over-encumbering the
 figures, they depict small test cases, which might not expose the
 full potential of the constructors.\par
see the \href{ForSyDe-Atom.html#naming_conv}{naming convention} rules
 on how to interpret, use and develop your own constructors.\par

\begin{haddockdesc}
\item[\begin{tabular}{@{}l}
farm22
\end{tabular}]\haddockbegindoc
\haddockbeginargs
\haddockdecltt{::} & \haddockdecltt{(a1
                                     -> a2
                                        -> (b1, b2))} & function (e.g. process) \\
                                                        \haddockdecltt{->} & \haddockdecltt{Vector a1} & first input vector \\
                                                                                                         \haddockdecltt{->} & \haddockdecltt{Vector a2} & second input vector \\
                                                                                                                                                          \haddockdecltt{->} & \haddockdecltt{(Vector b1, Vector b2)} & two output vectors \\
\end{tabulary}\par
\haddocktt{farm} is simply the \haddockid{Vector} instance of the skeletom \haddocktt{farm}
 pattern (see \haddockid{farm22}). If the function taken
 as argument is a process, then it creates a farm network of data
 parallel processes.\par
Constructors: \haddocktt{farm{\char 91}1-4{\char 93}{\char 91}1-4{\char 93}}.\par
\begin{quote}
{\haddockverb\begin{verbatim}
>>> let v1 = vector [1,2,3,4,5]

>>> S.farm21 (+) v1 v1
<2,4,6,8,10>

>>> let s1 = SY.signal [1,2,3,4,5]

>>> let v2 = vector [s1,s1,s1]

>>> S.farm11 (comb11 (+1)) v2
<{2,3,4,5,6},{2,3,4,5,6},{2,3,4,5,6}>

>>> S.farm21 (\x -> comb11 (+x)) v1 v2
<{2,3,4,5,6},{3,4,5,6,7},{4,5,6,7,8}>

\end{verbatim}}
\end{quote}(image: fig/eqs-skel-vector-farm.png)\par
           (image: fig/skel-vector-func-farm.png)
 (image: fig/skel-vector-func-farm-net.png)\par
           
\end{haddockdesc}
\begin{haddockdesc}
\item[\begin{tabular}{@{}l}
reduce\ ::\ (a\ ->\ a\ ->\ a)\ ->\ Vector\ a\ ->\ a
\end{tabular}]\haddockbegindoc
As the name suggests, it reduces a vector to an element based on
 an associative function. If the function is not associative, it can be treated like a pipeline.\par
\haddockid{Vector} instantiates the skeletons for both
 \haddockid{reduce} and \haddockid{reducei}.\par
\begin{quote}
{\haddockverb\begin{verbatim}
>>> let v1 = vector [1,2,3,4,5]

>>> S.reduce (+) v1
15

>>> let s1 = SY.signal [1,2,3,4,5]

>>> let s2 = SY.signal [10,10,10,10,10]

>>> let v2 = vector [s1,s1,s1]

>>> S.reduce (comb21 (+)) v2
{3,6,9,12,15}

>>> S.reducei (comb21 (+)) s2 v2
{13,16,19,22,25}

\end{verbatim}}
\end{quote}(image: fig/skel-vector-func-reducei.png)
 (image: fig/skel-vector-func-reducei-net.png)\par
           
\end{haddockdesc}
\begin{haddockdesc}
\item[\begin{tabular}{@{}l}
prefix\ ::\ (b\ ->\ b\ ->\ b)\ ->\ Vector\ b\ ->\ Vector\ b
\end{tabular}]\haddockbegindoc
\haddocktt{prefix} peforms the \emph{parallel prefix} operation on a vector.
 Equivalent process networks are constructed if processes are passed
 as arguments.\par
Similar to \haddockid{reduce} and \haddockid{reducei}, two versions \haddockid{prefix} and
 \haddocktt{prefixi} are provided.\par
\begin{quote}
{\haddockverb\begin{verbatim}
>>> let v1 = vector [1,2,3,4,5]

>>> prefix (+) v1
<15,14,12,9,5>

>>> let s1 = SY.signal [1,2,3,4,5]

>>> let s2 = SY.signal [10,10,10,10,10]

>>> let v2 = vector [s1,s1,s1]

>>> prefix (comb21 (+)) v2
<{3,6,9,12,15},{2,4,6,8,10},{1,2,3,4,5}>

>>> prefixi (comb21 (+)) s2 v2
<{13,16,19,22,25},{12,14,16,18,20},{11,12,13,14,15}>

\end{verbatim}}
\end{quote}(image: fig/eqs-skel-vector-prefix.png)\par
           (image: fig/skel-vector-func-prefix.png)
 (image: fig/skel-vector-func-prefix-net.png)\par
           (image: fig/skel-vector-func-prefixi.png)
 (image: fig/skel-vector-func-prefixi-net.png)\par
           
\end{haddockdesc}
\begin{haddockdesc}
\item[\begin{tabular}{@{}l}
suffix\ ::\ (b\ ->\ b\ ->\ b)\ ->\ Vector\ b\ ->\ Vector\ b
\end{tabular}]\haddockbegindoc
\haddocktt{suffix} peforms the \emph{parallel suffix} operation on a vector.
 Equivalent process networks are constructed if processes are passed
 as arguments.\par
Similar to \haddockid{reduce} and \haddockid{reducei}, two versions \haddockid{suffix} and
 \haddocktt{suffixi} are provided.\par
\begin{quote}
{\haddockverb\begin{verbatim}
>>> let v1 = vector [1,2,3,4,5]

>>> suffix (+) v1
<1,3,6,10,15>

>>> let s1 = SY.signal [1,2,3,4,5]

>>> let s2 = SY.signal [10,10,10,10,10]

>>> let v2 = vector [s1,s1,s1]

>>> suffix (comb21 (+)) v2
<{1,2,3,4,5},{2,4,6,8,10},{3,6,9,12,15}>

>>> suffixi (comb21 (+)) s2 v2
<{11,12,13,14,15},{12,14,16,18,20},{13,16,19,22,25}>

\end{verbatim}}
\end{quote}(image: fig/eqs-skel-vector-suffix.png)\par
           (image: fig/skel-vector-func-suffix.png)
 (image: fig/skel-vector-func-suffix-net.png)\par
           (image: fig/skel-vector-func-suffixi.png)
 (image: fig/skel-vector-func-suffixi-net.png)\par
           
\end{haddockdesc}
\begin{haddockdesc}
\item[\begin{tabular}{@{}l}
pipe
\end{tabular}]\haddockbegindoc
\haddockbeginargs
\haddockdecltt{::} & \haddockdecltt{Vector (a -> a)} & vector of functions \\
                                                       \haddockdecltt{->} & \haddockdecltt{a} & input \\
                                                                                                \haddockdecltt{->} & \haddockdecltt{a} & output \\
\end{tabulary}\par
\haddocktt{pipe} creates a pipeline of functions from a vector. \haddockid{pipe}
  simply instantiates the \haddockid{=<<=} atom whereas \haddocktt{pipeX} instantiate
  their omologi from the \haddocktt{ForSyDe.Atom.Skeleton} module (see
  \haddockid{pipe2}).\par
\textbf{OBS:} the pipelining is done in the order dictated by the
 function composition operator: from right to left.\par
Constructors: \haddocktt{pipe{\char 91}1-4{\char 93}}.\par
\begin{quote}
{\haddockverb\begin{verbatim}
>>> let v1 = vector [(+1),(+1),(+1)]

>>> S.pipe v1 1
4

>>> let s1 = SY.signal [1,2,3,4]

>>> let v2 = vector [1,2,3,4]

>>> S.pipe1 (\x -> comb11 (+x)) v2 s1
{11,12,13,14}

\end{verbatim}}
\end{quote}(image: fig/skel-vector-func-pipe.png)
 (image: fig/skel-vector-func-pipe-net.png)\par
           
\end{haddockdesc}
\begin{haddockdesc}
\item[\begin{tabular}{@{}l}
(=/=)\ ::\ Vector\ (a\ ->\ a)\ ->\ a\ ->\ Vector\ a
\end{tabular}]\haddockbegindoc
Infix operator for \haddockid{recur}.\par

\end{haddockdesc}
\begin{haddockdesc}
\item[\begin{tabular}{@{}l}
recur
\end{tabular}]\haddockbegindoc
\haddockbeginargs
\haddockdecltt{::} & \haddockdecltt{Vector (a -> a)} & vector of functions \\
                                                       \haddockdecltt{->} & \haddockdecltt{a} & input \\
                                                                                                \haddockdecltt{->} & \haddockdecltt{Vector a} & output  \\
\end{tabulary}\par
\haddocktt{recur} creates a systolic array from a vector of
 functions. Just like \haddockid{pipe} and \haddocktt{pipeX}, there exists a raw
 \haddockid{recur} version with an infix operator \haddockid{=/=}, and the enhanced
 \haddocktt{recurX} which is meant for systematic partial application of a
 function on an arbitrary number of vectors until the desired vector
 of functions is obtained.\par
Constructors: \haddocktt{(=/=)}, \haddocktt{recur}, \haddocktt{recuri}, \haddocktt{recur{\char 91}1-4{\char 93}{\char 91}1-4{\char 93}}.\par
\begin{quote}
{\haddockverb\begin{verbatim}
>>> let v1 = vector [(+1),(+1),(+1)]

>>> recur v1 1
<4,3,2>

>>> recuri v1 1
<4,3,2,1>

>>> let s1 = SY.signal [1,2,3,4]

>>> let v2 = vector [1,2,3,4]

>>> recur1 (\x -> comb11 (+x)) v2 s1
<{11,12,13,14},{10,11,12,13},{8,9,10,11},{5,6,7,8}>

\end{verbatim}}
\end{quote}(image: fig/eqs-skel-vector-recur.png)\par
           (image: fig/skel-vector-func-recur.png)
 (image: fig/skel-vector-func-recur-net.png)\par
           
\end{haddockdesc}
\begin{haddockdesc}
\item[\begin{tabular}{@{}l}
cascade2
\end{tabular}]\haddockbegindoc
\haddockbeginargs
\haddockdecltt{::} & \haddockdecltt{(a2
                                     -> a1
                                        -> a
                                           -> a
                                              -> a)} & \haddocktt{function41} which needs to be applied to \haddocktt{function21} \\
                                                       \haddockdecltt{->} & \haddockdecltt{Vector (Vector a2)} & fills in the first argument in the function above \\
                                                                                                                 \haddockdecltt{->} & \haddockdecltt{Vector (Vector a1)} & fills in the second argument in the function above \\
                                                                                                                                                                           \haddockdecltt{->} & \haddockdecltt{Vector a} & first input vector (e.g. of signals) \\
                                                                                                                                                                                                                           \haddockdecltt{->} & \haddockdecltt{Vector a} & second input vector (e.g. of signals) \\
                                                                                                                                                                                                                                                                           \haddockdecltt{->} & \haddockdecltt{Vector a} & output \\
\end{tabulary}\par
\haddocktt{cascade} creates a "cascading mesh" as a result of piping a
 vector into a vector of recur arrays. \par
Constructors: \haddocktt{cascade}, \haddocktt{cascade{\char 91}1-4{\char 93}}.\par
\begin{quote}
{\haddockverb\begin{verbatim}
>>> let v1 = vector [1,2,3,4]

>>> cascade (+) v1 v1
<238,119,49,14>

>>> let s1 = SY.signal [1,2,3,4]

>>> let vs = vector [s1, s1, s1]

>>> cascade (comb21 (+)) vs vs
<{20,40,60,80},{10,20,30,40},{4,8,12,16}>

>>> let vv = vector [vector [1,-1,1], vector [-1,1,-1], vector [1,-1,1] ]

>>> cascade1 (\x -> comb21 (\y z-> x*(y+z))) vv vs vs
<{16,32,48,64},{8,16,24,32},{-2,-4,-6,-8}>

\end{verbatim}}
\end{quote}(image: fig/eqs-skel-vector-cascade.png)\par
           (image: fig/skel-vector-func-cascade.png)
 (image: fig/skel-vector-func-cascade-net.png)\par
           
\end{haddockdesc}
\begin{haddockdesc}
\item[\begin{tabular}{@{}l}
mesh2
\end{tabular}]\haddockbegindoc
\haddockbeginargs
\haddockdecltt{::} & \haddockdecltt{(a2
                                     -> a1
                                        -> a
                                           -> a
                                              -> a)} & \haddocktt{function41} which needs to be applied to \haddocktt{function21} \\
                                                       \haddockdecltt{->} & \haddockdecltt{Vector (Vector a2)} & fills in the first argument in the function above \\
                                                                                                                 \haddockdecltt{->} & \haddockdecltt{Vector (Vector a1)} & fills in the second argument in the function above \\
                                                                                                                                                                           \haddockdecltt{->} & \haddockdecltt{Vector a} & first input vector (e.g. of signals) \\
                                                                                                                                                                                                                           \haddockdecltt{->} & \haddockdecltt{Vector a} & second input vector (e.g. of signals) \\
                                                                                                                                                                                                                                                                           \haddockdecltt{->} & \haddockdecltt{Vector (Vector a)} & output, a 2D vector \\
\end{tabulary}\par
\haddocktt{mesh} creates a 2D systolic array as a result of piping a vector
 into a vector of 1D systolic arrays.\par
Constructors: \haddocktt{mesh}, \haddocktt{mesh{\char 91}1-4{\char 93}}.\par
\begin{quote}
{\haddockverb\begin{verbatim}
>>> let v1 = vector [1,2,3,4]

>>> mesh (+) v1 v1
<<238,119,49,14>,<119,70,35,13>,<49,35,22,11>,<14,13,11,8>>

>>> let s1 = SY.signal [1,2,3,4]

>>> let vs = vector [s1, s1, s1]

>>> mesh (comb21 (+)) vs vs
<<{20,40,60,80},{10,20,30,40},{4,8,12,16}>,<{10,20,30,40},{6,12,18,24},{3,6,9,12}>,<{4,8,12,16},{3,6,9,12},{2,4,6,8}>>

>>> let vv = vector [vector [1,-1,1], vector [-1,1,-1], vector [1,-1,1]]

>>> mesh1 (\x -> comb21 (\y z-> x*(y+z))) vv vs vs
<<{16,32,48,64},{8,16,24,32},{-2,-4,-6,-8}>,<{8,16,24,32},{-6,-12,-18,-24},{-3,-6,-9,-12}>,<{-2,-4,-6,-8},{-3,-6,-9,-12},{2,4,6,8}>>

\end{verbatim}}
\end{quote}(image: fig/eqs-skel-vector-mesh.png)\par
           (image: fig/skel-vector-func-mesh.png)
 (image: fig/skel-vector-func-mesh-net.png)\par
           
\end{haddockdesc}
\subsection{Queries}
Queries return various information about a vector. They are
 also built as skeletons.\par

\begin{haddockdesc}
\item[\begin{tabular}{@{}l}
length\ ::\ Num\ p\ =>\ Vector\ a\ ->\ p
\end{tabular}]\haddockbegindoc
returns the number of elements in a value.\par
\begin{quote}
{\haddockverb\begin{verbatim}
>>> length $ vector [1,2,3,4,5]
5

\end{verbatim}}
\end{quote}(image: fig/eqs-skel-vector-length.png)\par
           
\end{haddockdesc}
\begin{haddockdesc}
\item[\begin{tabular}{@{}l}
index\ ::\ Vector\ a2\ ->\ Vector\ Integer
\end{tabular}]\haddockbegindoc
returns a vector with the indexes from another vector.\par
\begin{quote}
{\haddockverb\begin{verbatim}
>>> index $ vector [1,1,1,1,1,1,1]
<1,2,3,4,5,6,7>

\end{verbatim}}
\end{quote}
\end{haddockdesc}
\subsection{Generators}
Generators are specific applications of the \haddocktt{prefix} or
 \haddocktt{suffix} skeletons.\par

\begin{haddockdesc}
\item[\begin{tabular}{@{}l}
fanout\ ::\ t\ ->\ Vector\ t
\end{tabular}]\haddockbegindoc
\haddockid{fanout} repeats an element. As a process network it distributes
 the same value or signal to all the connected processes down the
 line. Depending on the target platform and the refinement decisions
 involved, it may be interpreted in the following implementations:\par
\begin{itemize}
\item
global or shared memory in case of a massively parallel platform
  (e.g. GPU)\par

\item
a (static) memory or cache location in memory-driven
  architectures (e.g. CPU)\par

\item
a fanout in case of a HDL system\par

\item
a broadcast in case of a distributed system\par

\end{itemize}

\end{haddockdesc}
\begin{haddockdesc}
\item[\begin{tabular}{@{}l}
fanoutn\ ::\ (Num\ t,\ Ord\ t)\ =>\ t\ ->\ a\ ->\ Vector\ a
\end{tabular}]\haddockbegindoc
\haddockid{fanoutn} is the same as \haddockid{fanout}, but the length of the result
 is also provided.\par
\begin{quote}
{\haddockverb\begin{verbatim}
>>> fanoutn 5 1
<1,1,1,1,1>

\end{verbatim}}
\end{quote}
\end{haddockdesc}
\begin{haddockdesc}
\item[\begin{tabular}{@{}l}
generate\ ::\ (Num\ t,\ Ord\ t)\ =>\ t\ ->\ (a\ ->\ a)\ ->\ a\ ->\ Vector\ a
\end{tabular}]\haddockbegindoc
\haddockid{generate} creates a vector based on a kernel function. It is
 just a restricted version of \haddockid{recur}.\par
\begin{quote}
{\haddockverb\begin{verbatim}
>>> generate 5 (+1) 1
<6,5,4,3,2>

\end{verbatim}}
\end{quote}(image: fig/eqs-skel-vector-generate.png)\par
           
\end{haddockdesc}
\begin{haddockdesc}
\item[\begin{tabular}{@{}l}
iterate\ ::\ (Num\ t,\ Ord\ t)\ =>\ t\ ->\ (a\ ->\ a)\ ->\ a\ ->\ Vector\ a
\end{tabular}]\haddockbegindoc
\haddockid{iterate} is a version of \haddockid{generate} which keeps the initial
 element as well. It is a restricted version of \haddocktt{recuri}.\par
\begin{quote}
{\haddockverb\begin{verbatim}
>>> iterate 5 (+1) 1
<5,4,3,2,1>

\end{verbatim}}
\end{quote}
\end{haddockdesc}
\subsection{Permutators}
Permutators perform operations on the very structure of
 vectors, and make heavy use of the vector constructors.\par

\begin{haddockdesc}
\item[\begin{tabular}{@{}l}
first\ ::\ Vector\ a\ ->\ a
\end{tabular}]\haddockbegindoc
Instance of \haddockid{first}\par
\begin{quote}
{\haddockverb\begin{verbatim}
>>> S.first $ vector [1,2,3,4,5]
1

\end{verbatim}}
\end{quote}
\end{haddockdesc}
\begin{haddockdesc}
\item[\begin{tabular}{@{}l}
last\ ::\ Vector\ a\ ->\ a
\end{tabular}]\haddockbegindoc
Instance of \haddockid{last}\par
\begin{quote}
{\haddockverb\begin{verbatim}
>>> S.last $ vector [1,2,3,4,5]
5

\end{verbatim}}
\end{quote}
\end{haddockdesc}
\begin{haddockdesc}
\item[\begin{tabular}{@{}l}
inits\ ::\ Vector\ a\ ->\ Vector\ (Vector\ a)
\end{tabular}]\haddockbegindoc
creates a vector of all the initial segments in a vector.\par
\begin{quote}
{\haddockverb\begin{verbatim}
>>> inits $ vector [1,2,3,4,5]
<<1>,<1,2>,<1,2,3>,<1,2,3,4>,<1,2,3,4,5>>

\end{verbatim}}
\end{quote}(image: fig/eqs-skel-vector-inits.png)\par
           (image: fig/skel-vector-comm-inits.png)
 (image: fig/skel-vector-comm-inits-net.png)\par
           
\end{haddockdesc}
\begin{haddockdesc}
\item[\begin{tabular}{@{}l}
tails\ ::\ Vector\ a\ ->\ Vector\ (Vector\ a)
\end{tabular}]\haddockbegindoc
creates a vector of all the final segments in a vector.\par
\begin{quote}
{\haddockverb\begin{verbatim}
>>> tails $ vector [1,2,3,4,5]
<<1,2,3,4,5>,<2,3,4,5>,<3,4,5>,<4,5>,<5>>

\end{verbatim}}
\end{quote}(image: fig/eqs-skel-vector-tails.png)\par
           (image: fig/skel-vector-comm-tails.png)
 (image: fig/skel-vector-comm-tails-net.png)\par
           
\end{haddockdesc}
\begin{haddockdesc}
\item[\begin{tabular}{@{}l}
init\ ::\ Vector\ a\ ->\ Vector\ a
\end{tabular}]\haddockbegindoc
Returns the initial segment of a vector.\par
\begin{quote}
{\haddockverb\begin{verbatim}
>>> init $ vector [1,2,3,4,5]
<1,2,3,4>

\end{verbatim}}
\end{quote}(image: fig/eqs-skel-vector-init.png)\par
           
\end{haddockdesc}
\begin{haddockdesc}
\item[\begin{tabular}{@{}l}
tail\ ::\ Vector\ a\ ->\ Vector\ a
\end{tabular}]\haddockbegindoc
Returns the tail of a vector.\par
\begin{quote}
{\haddockverb\begin{verbatim}
>>> tail $ vector [1,2,3,4,5]
<2,3,4,5>

\end{verbatim}}
\end{quote}(image: fig/eqs-skel-vector-tail.png)\par
           
\end{haddockdesc}
\begin{haddockdesc}
\item[\begin{tabular}{@{}l}
concat\ ::\ Vector\ (Vector\ a)\ ->\ Vector\ a
\end{tabular}]\haddockbegindoc
concatenates a vector of vectors.\par
\begin{quote}
{\haddockverb\begin{verbatim}
>>> concat $ vector [vector[1,2,3,4], vector[5,6,7]]
<1,2,3,4,5,6,7>

\end{verbatim}}
\end{quote}(image: fig/eqs-skel-vector-concat.png)\par
           
\end{haddockdesc}
\begin{haddockdesc}
\item[\begin{tabular}{@{}l}
reverse\ ::\ Vector\ a\ ->\ Vector\ a
\end{tabular}]\haddockbegindoc
reverses the elements in a vector.\par
\begin{quote}
{\haddockverb\begin{verbatim}
>>> reverse $ vector [1,2,3,4,5]
<5,4,3,2,1>

\end{verbatim}}
\end{quote}(image: fig/eqs-skel-vector-reverse.png)\par
           (image: fig/skel-vector-comm-reverse.png)
 (image: fig/skel-vector-comm-reverse-net.png)\par
           
\end{haddockdesc}
\begin{haddockdesc}
\item[\begin{tabular}{@{}l}
group\ ::\ Integer\ ->\ Vector\ a\ ->\ Vector\ (Vector\ a)
\end{tabular}]\haddockbegindoc
groups a vector into sub-vectors of \emph{n} elements.\par
\begin{quote}
{\haddockverb\begin{verbatim}
>>> group 3 $ vector [1,2,3,4,5,6,7,8]
<<1,2,3>,<4,5,6>,<7,8>>

\end{verbatim}}
\end{quote}(image: fig/eqs-skel-vector-group.png)\par
           (image: fig/skel-vector-comm-group.png)
 (image: fig/skel-vector-comm-group-net.png)\par
           
\end{haddockdesc}
\begin{haddockdesc}
\item[\begin{tabular}{@{}l}
shiftr\ ::\ Vector\ a\ ->\ a\ ->\ Vector\ a
\end{tabular}]\haddockbegindoc
right-shifts a vector with an element.\par
\begin{quote}
{\haddockverb\begin{verbatim}
>>> vector [1,2,3,4] `shiftr` 8
<8,1,2,3>

\end{verbatim}}
\end{quote}(image: fig/skel-vector-comm-shiftr.png)
 (image: fig/skel-vector-comm-shiftr-net.png)\par
           
\end{haddockdesc}
\begin{haddockdesc}
\item[\begin{tabular}{@{}l}
shiftl\ ::\ Vector\ a\ ->\ a\ ->\ Vector\ a
\end{tabular}]\haddockbegindoc
left-shifts a vector with an element.\par
\begin{quote}
{\haddockverb\begin{verbatim}
>>> vector [1,2,3,4] `shiftl` 8
<2,3,4,8>

\end{verbatim}}
\end{quote}(image: fig/skel-vector-comm-shiftl.png)
 (image: fig/skel-vector-comm-shiftl-net.png)\par
           
\end{haddockdesc}
\begin{haddockdesc}
\item[\begin{tabular}{@{}l}
rotr\ ::\ Vector\ a\ ->\ Vector\ a
\end{tabular}]\haddockbegindoc
rotates a vector to the right.\par
\begin{quote}
{\haddockverb\begin{verbatim}
>>> rotr $ vector [1,2,3,4]
<4,1,2,3>

\end{verbatim}}
\end{quote}(image: fig/skel-vector-comm-rotr.png)
 (image: fig/skel-vector-comm-rotr-net.png)\par
           
\end{haddockdesc}
\begin{haddockdesc}
\item[\begin{tabular}{@{}l}
rotl\ ::\ Vector\ a\ ->\ Vector\ a
\end{tabular}]\haddockbegindoc
rotates a vector to the left.\par
\begin{quote}
{\haddockverb\begin{verbatim}
>>> rotl $ vector [1,2,3,4]
<2,3,4,1>

\end{verbatim}}
\end{quote}(image: fig/skel-vector-comm-rotl.png)
 (image: fig/skel-vector-comm-rotl-net.png)\par
           
\end{haddockdesc}
\begin{haddockdesc}
\item[\begin{tabular}{@{}l}
take\ ::\ Integer\ ->\ Vector\ a\ ->\ Vector\ a
\end{tabular}]\haddockbegindoc
takes the first \emph{n} elements of a vector.\par
\begin{quote}
{\haddockverb\begin{verbatim}
>>> take 5 $ vector [1,2,3,4,5,6,7,8,9]
<1,2,3,4,5>

\end{verbatim}}
\end{quote}(image: fig/eqs-skel-vector-take.png)\par
           
\end{haddockdesc}
\begin{haddockdesc}
\item[\begin{tabular}{@{}l}
drop\ ::\ Integer\ ->\ Vector\ a\ ->\ Vector\ a
\end{tabular}]\haddockbegindoc
drops the first \emph{n} elements of a vector.\par
\begin{quote}
{\haddockverb\begin{verbatim}
>>> drop 5 $ vector [1,2,3,4,5,6,7,8,9]
<6,7,8,9>

\end{verbatim}}
\end{quote}(image: fig/eqs-skel-vector-drop.png)\par
           
\end{haddockdesc}
\begin{haddockdesc}
\item[\begin{tabular}{@{}l}
takeWhile\ ::\ (a\ ->\ Bool)\ ->\ Vector\ a\ ->\ Vector\ a
\end{tabular}]\haddockbegindoc
takes the first elements in a vector until the first element that
 does not fulfill a predicate.\par
\begin{quote}
{\haddockverb\begin{verbatim}
>>> takeWhile (<5) $ vector [1,2,3,4,5,6,7,8,9]
<1,2,3,4>

\end{verbatim}}
\end{quote}(image: fig/eqs-skel-vector-takewhile.png)\par
           
\end{haddockdesc}
\begin{haddockdesc}
\item[\begin{tabular}{@{}l}
filterIdx\ ::\ (Integer\ ->\ Bool)\ ->\ Vector\ a\ ->\ Vector\ a
\end{tabular}]\haddockbegindoc
returns a vector containing only the elements of another vector
 whose index satisfies a predicate.\par
\begin{quote}
{\haddockverb\begin{verbatim}
>>> filterIdx (\x -> x `mod` 3 == 0) $ vector [0,1,2,3,4,5,6,7,8,9]
<2,5,8>

\end{verbatim}}
\end{quote}(image: fig/eqs-skel-vector-filteridx.png)\par
           (image: fig/skel-vector-comm-filteridx.png)
 (image: fig/skel-vector-comm-filteridx-net.png)\par
           
\end{haddockdesc}
\begin{haddockdesc}
\item[\begin{tabular}{@{}l}
odds\ ::\ Vector\ a\ ->\ Vector\ a
\end{tabular}]\haddockbegindoc
(image: fig/eqs-skel-vector-odds.png)\par

\end{haddockdesc}
\begin{haddockdesc}
\item[\begin{tabular}{@{}l}
evens\ ::\ Vector\ a\ ->\ Vector\ a
\end{tabular}]\haddockbegindoc
(image: fig/eqs-skel-vector-evens.png)\par

\end{haddockdesc}
\begin{haddockdesc}
\item[\begin{tabular}{@{}l}
stride
\end{tabular}]\haddockbegindoc
\haddockbeginargs
\haddockdecltt{::} & \haddockdecltt{Integer} & first index \\
                                               \haddockdecltt{->} & \haddockdecltt{Integer} & stride length \\
                                                                                              \haddockdecltt{->} & \haddockdecltt{Vector a} & \\
                                                                                                                                              \haddockdecltt{->} & \haddockdecltt{Vector a} & \\
\end{tabulary}\par
does a stride-selection on a vector.\par
\begin{quote}
{\haddockverb\begin{verbatim}
>>> stride 1 3 $ vector [1,2,3,4,5,6,7,8,9]
<1,4,7>

\end{verbatim}}
\end{quote}(image: fig/eqs-skel-vector-stride.png)\par
           (image: fig/skel-vector-comm-inits.png)
 (image: fig/skel-vector-comm-inits-net.png)\par
           
\end{haddockdesc}
\begin{haddockdesc}
\item[\begin{tabular}{@{}l}
get\ ::\ Integer\ ->\ Vector\ a\ ->\ Maybe\ a
\end{tabular}]\haddockbegindoc
returns the \emph{n}-th element in a vector, or \haddocktt{Nothing} if \emph{n > l}.\par
\begin{quote}
{\haddockverb\begin{verbatim}
>>> get 3 $ vector [1,2,3,4,5]
Just 3

\end{verbatim}}
\end{quote}(image: fig/eqs-skel-vector-get.png)\par
           
\end{haddockdesc}
\begin{haddockdesc}
\item[\begin{tabular}{@{}l}
(<@)\ ::\ Vector\ a\ ->\ Integer\ ->\ Maybe\ a
\end{tabular}]\haddockbegindoc
the same as \haddockid{get} but with flipped arguments.\par

\end{haddockdesc}
\begin{haddockdesc}
\item[\begin{tabular}{@{}l}
(<@!)\ ::\ Vector\ p\ ->\ Integer\ ->\ p
\end{tabular}]\haddockbegindoc
unsafe version of \haddockid{<@>}. Throws an exception if \emph{n > l}.\par

\end{haddockdesc}
\begin{haddockdesc}
\item[\begin{tabular}{@{}l}
gather1
\end{tabular}]\haddockbegindoc
\haddockbeginargs
\haddockdecltt{::} & \haddockdecltt{Vector Integer} & vector of indexes \\
                                                      \haddockdecltt{->} & \haddockdecltt{Vector a} & input vector \\
                                                                                                      \haddockdecltt{->} & \haddockdecltt{Vector (Maybe a)} & \\
\end{tabulary}\par
selects the elements in a vector at the incexes contained by another vector.\par
The following versions of this skeleton are available, the number
 suggesting how many nested vectors it is operating upon: \haddocktt{gather{\char 91}1-5{\char 93}}\par
\begin{quote}
{\haddockverb\begin{verbatim}
>>> let ix = vector [vector [1,3,4], vector [3,5,1], vector [5,8,9]]

>>> let v = vector [11,12,13,14,15]

>>> gather2 ix v
<<Just 11,Just 13,Just 14>,<Just 13,Just 15,Just 11>,<Just 15,Nothing,Nothing>>

\end{verbatim}}
\end{quote}(image: fig/eqs-skel-vector-gather.png)\par
           (image: fig/skel-vector-comm-gather.png)
 (image: fig/skel-vector-comm-gather-net.png)\par
           
\end{haddockdesc}
\begin{haddockdesc}
\item[\begin{tabular}{@{}l}
(<@>)
\end{tabular}]\haddockbegindoc
\haddockbeginargs
\haddockdecltt{::} & \haddockdecltt{Vector a} & input vector \\
                                                \haddockdecltt{->} & \haddockdecltt{Vector Integer} & vector of indexes \\
                                                                                                      \haddockdecltt{->} & \haddockdecltt{Vector (Maybe a)} & \\
\end{tabulary}\par
the same as \haddockid{gather1} but with flipped arguments\par
The following versions of this skeleton are available, the number
 suggesting how many nested vectors it is operating upon.\par
\begin{quote}
{\haddockverb\begin{verbatim}
(<@>), (<<@>>), (<<<@>>>), (<<<<@>>>>), (<<<<<@>>>>>),\end{verbatim}}
\end{quote}

\end{haddockdesc}
\begin{haddockdesc}
\item[\begin{tabular}{@{}l}
replace\ ::\ Integer\ ->\ p\ ->\ Vector\ p\ ->\ Vector\ p
\end{tabular}]\haddockbegindoc
replaces the \emph{n}-th element in a vector with another.\par
\begin{quote}
{\haddockverb\begin{verbatim}
>>> replace 5 15 $ vector [1,2,3,4,5,6,7,8,9]
<1,2,3,4,15,6,7,8,9>

\end{verbatim}}
\end{quote}(image: fig/eqs-skel-vector-replace.png)\par
           (image: fig/skel-vector-comm-replace.png)
 (image: fig/skel-vector-comm-replace-net.png)\par
           
\end{haddockdesc}
\begin{haddockdesc}
\item[\begin{tabular}{@{}l}
scatter\ ::\ Vector\ Integer\ ->\ Vector\ p\ ->\ Vector\ p\ ->\ Vector\ p
\end{tabular}]\haddockbegindoc
scatters the elements in a vector based on the indexes contained by another vector.\par
\begin{quote}
{\haddockverb\begin{verbatim}
>>> scatter (vector [2,4,5]) (vector [0,0,0,0,0,0,0,0]) (vector [1,1,1])
<0,1,0,1,1,0,0,0>

\end{verbatim}}
\end{quote}(image: fig/eqs-skel-vector-scatter.png)\par
           (image: fig/skel-vector-comm-scatter.png)
 (image: fig/skel-vector-comm-scatter-net.png)\par
           
\end{haddockdesc}
\begin{haddockdesc}
\item[\begin{tabular}{@{}l}
bitrev\ ::\ Vector\ a\ ->\ Vector\ a
\end{tabular}]\haddockbegindoc
performs a bit-reverse permutation.\par
(image: fig/skel-vector-comm-bitrev.png)
 (image: fig/skel-vector-comm-bitrev-net.png)\par
\begin{quote}
{\haddockverb\begin{verbatim}
>>> bitrev $ vector ["000","001","010","011","100","101","110","111"]
<"111","011","101","001","110","010","100","000">

\end{verbatim}}
\end{quote}
\end{haddockdesc}
\begin{haddockdesc}
\item[\begin{tabular}{@{}l}
duals\ ::\ Vector\ b2\ ->\ (Vector\ b2,\ Vector\ b2)
\end{tabular}]\haddockbegindoc
splits a vector in two equal parts.\par
\begin{quote}
{\haddockverb\begin{verbatim}
>>> duals $ vector [1,2,3,4,5,6,7]
(<1,2,3>,<4,5,6>)

\end{verbatim}}
\end{quote}
\end{haddockdesc}
\begin{haddockdesc}
\item[\begin{tabular}{@{}l}
unduals\ ::\ Vector\ a\ ->\ Vector\ a\ ->\ Vector\ a
\end{tabular}]\haddockbegindoc
concatenates a previously split vector. See also \haddockid{duals}\par

\end{haddockdesc}
\subsection{Interfaces}
\begin{haddockdesc}
\item[\begin{tabular}{@{}l}
zipx
\end{tabular}]\haddockbegindoc
\haddockbeginargs
\haddockdecltt{::} & MoC e \\
                     \haddockdecltt{=>} & \haddockdecltt{Vector ((Vector a
                                                                  -> Vector a
                                                                     -> Vector a)
                                                                 -> Fun e (Vector a) (Fun e (Vector a) (Ret e (Vector a))))} & vector of MoC-specific context wrappers for the function
 \haddockid{<++>} \\
                                                                                                                               \haddockdecltt{->} & \haddockdecltt{Vector (Stream (e a))} & input vector of signals \\
                                                                                                                                                                                            \haddockdecltt{->} & \haddockdecltt{Stream (e (Vector a))} & output signal of vectors \\
\end{tabulary}\par
\haddockid{zipx} is a template skeleton for "zipping" a vector of
 signals. It synchronizes all signals (of the same MoC) in a vector
 and outputs one signal with vectors of the synced values. For each
 signal in the input vector it requires a function which
 \emph{translates} a partition of events (see \haddocktt{ForSyDe.Atom.MoC}) into
 sub-vectors.\par
There exist helper intances of the \haddockid{zipx} skeleton interface for
 all supported MoCs.\par
(image: fig/eqs-skel-vector-zipx.png)
 (image: fig/skel-vector-comm-zipx.png)\par

\end{haddockdesc}
\begin{haddockdesc}
\item[\begin{tabular}{@{}l}
unzipx\ ::\ MoC\ e\ =>\\\ \ \ \ \ \ \ \ \ \ (Vector\ a\ ->\ Vector\ (Ret\ e\ a))\\\ \ \ \ \ \ \ \ \ \ ->\ Integer\ ->\ Stream\ (e\ (Vector\ a))\ ->\ Vector\ (Stream\ (e\ a))
\end{tabular}]\haddockbegindoc
\haddockid{unzipx} is a template skeleton to unzip a signal carrying
 vectors into a vector of multiple signals. It required a function
 that \emph{splits} a vector of values into a vector of event partitions
 belonging to output signals. Unlike \haddockid{zipx}, it also requires the
 number of output signals. The reason for this is that it is
 impossible to determine the length of the output vector without
 "sniffing" the content of the input events, which is out of the
 scope of skeletons and may lead to unsafe behavior. The length of
 the output vector is needed in order to avoid infinite recurrence.\par
There exist helper intances of the \haddockid{unzipx} skeleton interface for
 all supported MoCs.\par
(image: fig/eqs-skel-vector-unzipx.png)
 (image: fig/skel-vector-comm-unzipx.png)\par

\end{haddockdesc}