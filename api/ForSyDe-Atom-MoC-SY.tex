\haddockmoduleheading{ForSyDe.Atom.MoC.SY}
\label{module:ForSyDe.Atom.MoC.SY}
\haddockbeginheader
{\haddockverb\begin{verbatim}
module ForSyDe.Atom.MoC.SY (
    SY(SY, val),  Signal,  unit2,  signal,  readSignal,  delay,  comb22, 
    reconfig22,  constant2,  generate2,  stated22,  state22,  moore22, 
    mealy22,  when,  when',  is,  whenPresent,  filter,  filter',  fill,  hold, 
    reactAbst2,  toDE2,  toSDF2,  zipx,  unzipx,  unzipx'
  ) where\end{verbatim}}
\haddockendheader

The \haddocktt{SY} library implements the atoms holding the sematics for the
 synchronous computation model. It also provides a set of helpers
 for properly instantiating process network patterns as process
 constructors.\par
\textbf{IMPORTANT!!!}
 see the \href{ForSyDe-Atom.html#naming_conv}{naming convention} rules
 on how to interpret, use and develop your own constructors.\par

\section{Synchronous (\haddocktt{SY}) event}
According to \href{ForSyDe-Atom.html#lee98}{[Lee98]}, "two events
 are synchronous if they have the same tag, and two signals are
 synchronous if all events in one signal are synchronous to an
 event in the second signal and vice-versa. A system is
 synchronous if every signals in the system is synchronous to all
 the other signals in the system."\par
The synchronous (\haddocktt{SY}) MoC defines no notion of physical time,
 its tag system suggesting in fact the precedence among events. To
 further demystify its mechanisms, we can formulate the following
 proposition:\par
\begin{description}
\item[The SY MoC] is abstracting the execution semantics of a system
 where computation is assumed to perform instantaneously (with
 zero delay), at certain synchronization points, when data is
 assumed to be available.
\end{description}Below is a \emph{possible} behavior in time of the input and the
 output signals of a SY process, to illustrate these semantics:\par
                 (image: fig/moc-sy-example.png)\par
                 Implementing the SY tag system is straightforward if we consider
 the synchronous \haddockid{Signal} as an infinite list. In this case the
 tags are implicitly defined by the position of events in a
 signal: \emph{t₀} would correspond with the event at the head of
 a signal \emph{t₁} with the following event, etc... The only
 explicit parameter passed to a SY event constructor is the value
 it carries ∈ \emph{Vₑ}. As such, we can state the
 following particularities:\par
                 \begin{enumerate}
                 \item 
                 tags are implicit from the position in the
 \haddockid{Stream}, thus they are completely
 ignored in the type constructor.\par
                 
                 \item 
                 the type constructor wraps only a value\par
                 
                 \item 
                 being a \emph{timed MoC}, the order between events is total
 \href{ForSyDe-Atom.html#lee98}{[Lee98]}.\par
                 
                 \item 
                 from the previous statement we can conclude that there is no
 need for an \href{ForSyDe-Atom-MoC.html#context}{execution context}
 and we can ignore the formatting of functions in
 \haddocktt{ForSyDe.Atom.MoC}, thus we can safely assume:
 (image: fig/eqs-moc-timed-context.png)\par
                 
                 \end{enumerate}
                 
\begin{haddockdesc}
\item[\begin{tabular}{@{}l}
newtype\ SY\ a
\end{tabular}]\haddockbegindoc
\haddockbeginconstrs
\haddockdecltt{=} & \haddockdecltt{SY} & \\
                    \haddockdecltt{val :: a} &
\end{tabulary}\par
\haddockbeginconstrs
\end{tabulary}\par
The SY event. It identifies a synchronous signal.\par

\end{haddockdesc}
\begin{haddockdesc}
\item[\begin{tabular}{@{}l}
instance\ Functor\ SY
\end{tabular}]\haddockbegindoc
Allows for mapping of functions on a SY event.\par

\end{haddockdesc}
\begin{haddockdesc}
\item[\begin{tabular}{@{}l}
instance\ Applicative\ SY
\end{tabular}]\haddockbegindoc
Allows for lifting functions on a pair of SY events.\par

\end{haddockdesc}
\begin{haddockdesc}
\item[\begin{tabular}{@{}l}
instance\ MoC\ SY
\end{tabular}]\haddockbegindoc
Implenents the execution and synchronization semantics for the SY
 MoC through its atoms.\par

\end{haddockdesc}
\begin{haddockdesc}
\item[\begin{tabular}{@{}l}
instance\ Read\ a\ =>\ Read\ (SY\ a)
\end{tabular}]\haddockbegindoc
Reads the value wrapped\par

\end{haddockdesc}
\begin{haddockdesc}
\item[\begin{tabular}{@{}l}
instance\ Show\ a\ =>\ Show\ (SY\ a)
\end{tabular}]\haddockbegindoc
Shows the value wrapped\par

\end{haddockdesc}
\begin{haddockdesc}
\item[\begin{tabular}{@{}l}
instance\ Plottable\ a\ =>\ Plot\ (Signal\ a)
\end{tabular}]\haddockbegindoc
\haddockid{SY} signals.\par

\end{haddockdesc}
\begin{haddockdesc}
\item[\begin{tabular}{@{}l}
instance\ type\ Ret\ SY\ b\ =\ b\\instance\ type\ Fun\ SY\ a\ b\ =\ a\ ->\ b
\end{tabular}]
\end{haddockdesc}
\section{Aliases {\char '46} utilities}
A set of type synonyms and utilities are provided for
 convenience. The API type signatures will feature these aliases
 to hide the cumbersome construction of atoms and patters as seen
 in \haddocktt{ForSyDe.Atom.MoC}.\par

\begin{haddockdesc}
\item[\begin{tabular}{@{}l}
type\ Signal\ a\ =\ Stream\ (SY\ a)
\end{tabular}]\haddockbegindoc
Type synonym for a SY signal, i.e. "an ordered stream of SY
 events"\par

\end{haddockdesc}
\begin{haddockdesc}
\item[\begin{tabular}{@{}l}
unit2\ ::\ (a1,\ a2)\ ->\ (Signal\ a1,\ Signal\ a2)
\end{tabular}]\haddockbegindoc
Wraps a (tuple of) value(s) into the equivalent unit signal(s).\par
Helpers: \haddocktt{unit}, \haddocktt{unit2}, \haddocktt{unit3}, \haddocktt{unit4}.\par

\end{haddockdesc}
\begin{haddockdesc}
\item[\begin{tabular}{@{}l}
signal\ ::\ {\char 91}a{\char 93}\ ->\ Signal\ a
\end{tabular}]\haddockbegindoc
Transforms a list of values into a SY signal.\par

\end{haddockdesc}
\begin{haddockdesc}
\item[\begin{tabular}{@{}l}
readSignal\ ::\ Read\ a\ =>\ String\ ->\ Signal\ a
\end{tabular}]\haddockbegindoc
Reads a signal from a string. Like with the \haddocktt{read} function from
 \haddocktt{Prelude}, you must specify the tipe of the signal.\par
\begin{quote}
{\haddockverb\begin{verbatim}
>>> readSignal "{1,2,3,4,5}" :: Signal Int
{1,2,3,4,5}

\end{verbatim}}
\end{quote}
\end{haddockdesc}
\section{\haddocktt{SY} process constuctors}
These SY process constructors are basically specific
 instantiations of the patterns of atoms defined in
 \haddocktt{ForSyDe.Atom.MoC}. Some are also wrapping functions in an
 extended behavioural model.\par

\subsection{Simple}
\begin{haddockdesc}
\item[\begin{tabular}{@{}l}
delay
\end{tabular}]\haddockbegindoc
\haddockbeginargs
\haddockdecltt{::} & \haddockdecltt{a} & initial value \\
                                         \haddockdecltt{->} & \haddockdecltt{Signal a} & input signal \\
                                                                                         \haddockdecltt{->} & \haddockdecltt{Signal a} & output signal \\
\end{tabulary}\par
The \haddocktt{delay} process "delays" a signal with one
 event. Instantiates the \haddockid{delay} pattern.\par
\begin{quote}
{\haddockverb\begin{verbatim}
>>> let s = signal [1,2,3,4,5]

>>> delay 0 s
{0,1,2,3,4,5}

\end{verbatim}}
\end{quote}(image: fig/moc-sy-pattern-delay.png)\par
           
\end{haddockdesc}
\begin{haddockdesc}
\item[\begin{tabular}{@{}l}
comb22
\end{tabular}]\haddockbegindoc
\haddockbeginargs
\haddockdecltt{::} & \haddockdecltt{(a1
                                     -> a2 -> (b1, b2))} & function on values \\
                                                           \haddockdecltt{->} & \haddockdecltt{Signal a1} & first input signal \\
                                                                                                            \haddockdecltt{->} & \haddockdecltt{Signal a2} & second input signal \\
                                                                                                                                                             \haddockdecltt{->} & \haddockdecltt{(Signal b1, Signal b2)} & two output signals \\
\end{tabulary}\par
\haddocktt{comb} processes map combinatorial functions on signals and take
 care of synchronization between input signals. It instantiates the
 \haddocktt{comb} pattern (see \haddockid{comb22}).\par
Constructors: \haddocktt{comb{\char 91}1-4{\char 93}{\char 91}1-4{\char 93}}.\par
\begin{quote}
{\haddockverb\begin{verbatim}
>>> let s1 = signal [1..]

>>> let s2 = signal [1,1,1,1,1]

>>> comb11 (+1) s2
{2,2,2,2,2}

>>> comb22 (\a b-> (a+b,a-b)) s1 s2
({2,3,4,5,6},{0,1,2,3,4})

\end{verbatim}}
\end{quote}(image: fig/moc-sy-pattern-comb.png)\par
           
\end{haddockdesc}
\begin{haddockdesc}
\item[\begin{tabular}{@{}l}
reconfig22
\end{tabular}]\haddockbegindoc
\haddockbeginargs
\haddockdecltt{::} & \haddockdecltt{Signal (a1
                                            -> a2
                                               -> (b1, b2))} & signal carrying functions \\
                                                               \haddockdecltt{->} & \haddockdecltt{Signal a1} & first input signal carrying arguments \\
                                                                                                                \haddockdecltt{->} & \haddockdecltt{Signal a2} & second input signal carrying arguments \\
                                                                                                                                                                 \haddockdecltt{->} & \haddockdecltt{(Signal b1, Signal b2)} & two output signals \\
\end{tabulary}\par
\haddocktt{reconfig} creates an synchronous adaptive process where the
 first signal carries functions and the other carry the
 arguments. It instantiates the \haddocktt{reconfig} atom pattern (see
 \haddockid{reconfig22}).\par
Constructors: \haddocktt{reconfig{\char 91}1-4{\char 93}{\char 91}1-4{\char 93}}.\par
\begin{quote}
{\haddockverb\begin{verbatim}
>>> let sf = signal [(+1),(*2),(+1),(*2),(+1),(*2),(+1)]

>>> let s1 = signal [1..]

>>> reconfig11 sf s1
{2,4,4,8,6,12,8}

\end{verbatim}}
\end{quote}(image: fig/moc-sy-pattern-reconfig.png)\par
           
\end{haddockdesc}
\begin{haddockdesc}
\item[\begin{tabular}{@{}l}
constant2
\end{tabular}]\haddockbegindoc
\haddockbeginargs
\haddockdecltt{::} & \haddockdecltt{(b1, b2)} & values to be repeated \\
                                                \haddockdecltt{->} & \haddockdecltt{(Signal b1, Signal b2)} & generated signals \\
\end{tabulary}\par
A signal generator which keeps a value constant. It
 is actually an instantiation of the \haddocktt{stated0X} constructor
 (check \haddockid{stated22}).\par
Constructors: \haddocktt{constant{\char 91}1-4{\char 93}}.\par
\begin{quote}
{\haddockverb\begin{verbatim}
>>> let (s1, s2) = constant2 (1,2)

>>> takeS 3 s1
{1,1,1}

>>> takeS 5 s2
{2,2,2,2,2}

\end{verbatim}}
\end{quote}(image: fig/moc-sy-pattern-constant.png)\par
           
\end{haddockdesc}
\begin{haddockdesc}
\item[\begin{tabular}{@{}l}
generate2
\end{tabular}]\haddockbegindoc
\haddockbeginargs
\haddockdecltt{::} & \haddockdecltt{(b1
                                     -> b2
                                        -> (b1, b2))} & function to generate next value \\
                                                        \haddockdecltt{->} & \haddockdecltt{(b1, b2)} & kernel values \\
                                                                                                        \haddockdecltt{->} & \haddockdecltt{(Signal b1, Signal b2)} & generated signals \\
\end{tabulary}\par
A signal generator based on a function and a kernel value. It
 is actually an instantiation of the \haddocktt{stated0X} constructor
 (check \haddockid{stated22}).\par
Constructors: \haddocktt{generate{\char 91}1-4{\char 93}}.\par
\begin{quote}
{\haddockverb\begin{verbatim}
>>> let (s1,s2) = generate2 (\a b -> (a+1,b+2)) (1,2)

>>> takeS 5 s1
{1,2,3,4,5}

>>> takeS 7 s2
{2,4,6,8,10,12,14}

\end{verbatim}}
\end{quote}(image: fig/moc-sy-pattern-generate.png)\par
           
\end{haddockdesc}
\begin{haddockdesc}
\item[\begin{tabular}{@{}l}
stated22
\end{tabular}]\haddockbegindoc
\haddockbeginargs
\haddockdecltt{::} & \haddockdecltt{(b1
                                     -> b2
                                        -> a1
                                           -> a2
                                              -> (b1, b2))} & next state function \\
                                                              \haddockdecltt{->} & \haddockdecltt{(b1, b2)} & initial state values \\
                                                                                                              \haddockdecltt{->} & \haddockdecltt{Signal a1} & first input signal \\
                                                                                                                                                               \haddockdecltt{->} & \haddockdecltt{Signal a2} & second input signal \\
                                                                                                                                                                                                                \haddockdecltt{->} & \haddockdecltt{(Signal b1, Signal b2)} & output signals \\
\end{tabulary}\par
\haddocktt{stated} is a state machine without an output decoder. It is an
 instantiation of the \haddocktt{state} MoC constructor
 (see \haddockid{stated22}).\par
Constructors: \haddocktt{stated{\char 91}1-4{\char 93}{\char 91}1-4{\char 93}}.\par
\begin{quote}
{\haddockverb\begin{verbatim}
>>> let s1 = signal [1,2,3,4,5]

>>> stated11 (+) 1 s1
{1,2,4,7,11,16}

\end{verbatim}}
\end{quote}(image: fig/moc-sy-pattern-stated.png)\par
           
\end{haddockdesc}
\begin{haddockdesc}
\item[\begin{tabular}{@{}l}
state22
\end{tabular}]\haddockbegindoc
\haddockbeginargs
\haddockdecltt{::} & \haddockdecltt{(b1
                                     -> b2
                                        -> a1
                                           -> a2
                                              -> (b1, b2))} & next state function \\
                                                              \haddockdecltt{->} & \haddockdecltt{(b1, b2)} & initial state values \\
                                                                                                              \haddockdecltt{->} & \haddockdecltt{Signal a1} & first input signal \\
                                                                                                                                                               \haddockdecltt{->} & \haddockdecltt{Signal a2} & second input signal \\
                                                                                                                                                                                                                \haddockdecltt{->} & \haddockdecltt{(Signal b1, Signal b2)} & output signals \\
\end{tabulary}\par
\haddocktt{state} is a state machine without an output decoder. It is an
 instantiation of the \haddocktt{stated} MoC constructor
 (see \haddockid{state22}).\par
Constructors: \haddocktt{state{\char 91}1-4{\char 93}{\char 91}1-4{\char 93}}.\par
\begin{quote}
{\haddockverb\begin{verbatim}
>>> let s1 = signal [1,2,3,4,5]

>>> state11 (+) 1 s1
{2,4,7,11,16}

\end{verbatim}}
\end{quote}(image: fig/moc-sy-pattern-state.png)\par
           
\end{haddockdesc}
\begin{haddockdesc}
\item[\begin{tabular}{@{}l}
moore22
\end{tabular}]\haddockbegindoc
\haddockbeginargs
\haddockdecltt{::} & \haddockdecltt{(st
                                     -> a1
                                        -> a2 -> st)} & next state function \\
                                                        \haddockdecltt{->} & \haddockdecltt{(st
                                                                                             -> (b1, b2))} & output decoder \\
                                                                                                             \haddockdecltt{->} & \haddockdecltt{st} & initial state \\
                                                                                                                                                       \haddockdecltt{->} & \haddockdecltt{Signal a1} & \\
                                                                                                                                                                                                        \haddockdecltt{->} & \haddockdecltt{Signal a2} & \\
                                                                                                                                                                                                                                                         \haddockdecltt{->} & \haddockdecltt{(Signal b1, Signal b2)} & \\
\end{tabulary}\par
\haddocktt{moore} processes model Moore state machines. It is an
 instantiation of the \haddocktt{moore} MoC constructor
 (see \haddockid{moore22}).\par
Constructors: \haddocktt{moore{\char 91}1-4{\char 93}{\char 91}1-4{\char 93}}.\par
\begin{quote}
{\haddockverb\begin{verbatim}
>>> let s1 = signal [1,2,3,4,5]

>>> moore11 (+) (+1) 1 s1
{2,3,5,8,12,17}

\end{verbatim}}
\end{quote}(image: fig/moc-sy-pattern-moore.png)\par
           
\end{haddockdesc}
\begin{haddockdesc}
\item[\begin{tabular}{@{}l}
mealy22
\end{tabular}]\haddockbegindoc
\haddockbeginargs
\haddockdecltt{::} & \haddockdecltt{(st
                                     -> a1
                                        -> a2 -> st)} & next state function \\
                                                        \haddockdecltt{->} & \haddockdecltt{(st
                                                                                             -> a1
                                                                                                -> a2
                                                                                                   -> (b1, b2))} & outpt decoder \\
                                                                                                                   \haddockdecltt{->} & \haddockdecltt{st} & initial state \\
                                                                                                                                                             \haddockdecltt{->} & \haddockdecltt{Signal a1} & \\
                                                                                                                                                                                                              \haddockdecltt{->} & \haddockdecltt{Signal a2} & \\
                                                                                                                                                                                                                                                               \haddockdecltt{->} & \haddockdecltt{(Signal b1, Signal b2)} & \\
\end{tabulary}\par
\haddocktt{mealy} processes model Mealy state machines. It is an
 instantiation of the \haddocktt{mealy} MoC constructor
 (see \haddockid{mealy22}).\par
Constructors: \haddocktt{mealy{\char 91}1-4{\char 93}{\char 91}1-4{\char 93}}.\par
\begin{quote}
{\haddockverb\begin{verbatim}
>>> let s1 = signal [1,2,3,4,5]

>>> mealy11 (+) (-) 1 s1
{0,0,1,3,6}

\end{verbatim}}
\end{quote}(image: fig/moc-sy-pattern-mealy.png)\par
           
\end{haddockdesc}
\subsection{Predicate behavior}
These processes manipulate the behavior of a signal based on
 predicates on their status.\par

\begin{haddockdesc}
\item[\begin{tabular}{@{}l}
when
\end{tabular}]\haddockbegindoc
\haddockbeginargs
\haddockdecltt{::} & \haddockdecltt{Signal (AbstExt Bool)} & Signal of predicates \\
                                                             \haddockdecltt{->} & \haddockdecltt{Signal (AbstExt a)} & Input signal \\
                                                                                                                       \haddockdecltt{->} & \haddockdecltt{Signal (AbstExt a)} & Output signal \\
\end{tabulary}\par
This process predicates the existence of values in a signal based
 on a signal of boolean values (conditions). It is similar to the
 \haddocktt{when} construct in the synchronous language Lustre
 \href{ForSyDe-Atom.html#halbwachs91}{[Halbwachs91]}, based on which clock
 calculus can be performed.\par
\textbf{OBS:} this process assumes that all signals carry
 absent-extended values, which is appropriate in describing
 multi-clock systems. For a version which inputs signals of
 non-extended values, check \haddockid{when'}.\par
\begin{quote}
{\haddockverb\begin{verbatim}
>>> let s1   = (signal . map Prst) [1,2,3,4,5]

>>> let pred = (signal . map Prst) [False,False,False,True,True]

>>> when pred s1
{⟂,⟂,⟂,4,5}

\end{verbatim}}
\end{quote}(image: fig/moc-sy-pattern-when.png)\par
           
\end{haddockdesc}
\begin{haddockdesc}
\item[\begin{tabular}{@{}l}
when'
\end{tabular}]\haddockbegindoc
\haddockbeginargs
\haddockdecltt{::} & \haddockdecltt{Signal Bool} & Signal of predicates \\
                                                   \haddockdecltt{->} & \haddockdecltt{Signal a} & Input signal \\
                                                                                                   \haddockdecltt{->} & \haddockdecltt{Signal (AbstExt a)} & Output signal \\
\end{tabulary}\par
Same as \haddockid{when} but inputs signals of non-extended values.\par
\begin{quote}
{\haddockverb\begin{verbatim}
>>> let s1   = signal [1,2,3,4,5]

>>> let pred = signal [False,False,False,True,True]

>>> when' pred s1
{⟂,⟂,⟂,4,5}

\end{verbatim}}
\end{quote}
\end{haddockdesc}
\begin{haddockdesc}
\item[\begin{tabular}{@{}l}
is\ ::\ Signal\ (AbstExt\ a)\ ->\ (a\ ->\ Bool)\ ->\ Signal\ (AbstExt\ Bool)
\end{tabular}]\haddockbegindoc
Simple wrapper for applying a predicate function on a signal of
 absent-extended events.\par
\begin{quote}
{\haddockverb\begin{verbatim}
>>> let s1   = signal $ map Prst [1,2,3,4,5]

>>> s1 `is` (>3)
{False,False,False,True,True}

\end{verbatim}}
\end{quote}
\end{haddockdesc}
\begin{haddockdesc}
\item[\begin{tabular}{@{}l}
whenPresent\ ::\ Signal\ (AbstExt\ a1)\\\ \ \ \ \ \ \ \ \ \ \ \ \ \ \ ->\ Signal\ (AbstExt\ a2)\ ->\ Signal\ (AbstExt\ a2)
\end{tabular}]\haddockbegindoc
Same as \haddockid{when} but triggering the output events merely based on
 the presence of the first input rather than a predicate function.\par
\begin{quote}
{\haddockverb\begin{verbatim}
>>> let s1   = signal $ map Prst [1,2,3,4,5]

>>> let sp   = signal [Prst 1, Prst 1, Abst, Prst 1, Abst]

>>> whenPresent sp s1
{1,2,⟂,4,⟂}

\end{verbatim}}
\end{quote}
\end{haddockdesc}
\begin{haddockdesc}
\item[\begin{tabular}{@{}l}
filter
\end{tabular}]\haddockbegindoc
\haddockbeginargs
\haddockdecltt{::} & \haddockdecltt{(a -> Bool)} & Predicate function \\
                                                   \haddockdecltt{->} & \haddockdecltt{Signal (AbstExt a)} & Input signal \\
                                                                                                             \haddockdecltt{->} & \haddockdecltt{Signal (AbstExt a)} & Output signal \\
\end{tabulary}\par
Filters out values to \haddockid{Abst} if they do not fulfill a predicate
 function.\par
\textbf{OBS:} this process assumes that all signals carry
 absent-extended values, which is appropriate in describing
 multi-clock systems. For a version which inputs signals of
 non-extended values, check \haddockid{filter'}.\par
\begin{quote}
{\haddockverb\begin{verbatim}
>>> let s1   = (signal . map Prst) [1,2,3,4,5]

>>> filter (>3) s1
{⟂,⟂,⟂,4,5}

\end{verbatim}}
\end{quote}(image: fig/moc-sy-pattern-filter.png)\par
           
\end{haddockdesc}
\begin{haddockdesc}
\item[\begin{tabular}{@{}l}
filter'
\end{tabular}]\haddockbegindoc
\haddockbeginargs
\haddockdecltt{::} & \haddockdecltt{(a -> Bool)} & Predicate function \\
                                                   \haddockdecltt{->} & \haddockdecltt{Signal a} & Input signal \\
                                                                                                   \haddockdecltt{->} & \haddockdecltt{Signal (AbstExt a)} & Output signal \\
\end{tabulary}\par
Same as \haddockid{filter} but inputs signals of non-extended values.\par
\begin{quote}
{\haddockverb\begin{verbatim}
>>> let s1   = signal [1,2,3,4,5]

>>> filter' (>3) s1
{⟂,⟂,⟂,4,5}

\end{verbatim}}
\end{quote}
\end{haddockdesc}
\begin{haddockdesc}
\item[\begin{tabular}{@{}l}
fill
\end{tabular}]\haddockbegindoc
\haddockbeginargs
\haddockdecltt{::} & \haddockdecltt{a} & Value to fill with \\
                                         \haddockdecltt{->} & \haddockdecltt{Signal (AbstExt a)} & Input \\
                                                                                                   \haddockdecltt{->} & \haddockdecltt{Signal a} & Output \\
\end{tabulary}\par
Fills absent events with a pre-defined value.\par
\begin{quote}
{\haddockverb\begin{verbatim}
>>> let s1   = signal [Abst, Abst, Prst 1, Prst 2, Abst, Prst 3]

>>> fill 0 s1
{0,0,1,2,0,3}

\end{verbatim}}
\end{quote}(image: fig/moc-sy-pattern-fill.png)\par
           
\end{haddockdesc}
\begin{haddockdesc}
\item[\begin{tabular}{@{}l}
hold
\end{tabular}]\haddockbegindoc
\haddockbeginargs
\haddockdecltt{::} & \haddockdecltt{a} & Value to fill with in case there was no previous value \\
                                         \haddockdecltt{->} & \haddockdecltt{Signal (AbstExt a)} & Input \\
                                                                                                   \haddockdecltt{->} & \haddockdecltt{Signal a} & Output \\
\end{tabulary}\par
Similar to \haddockid{fill}, but holds the last non-absent value if there
 was one. It implements a \haddocktt{state} pattern (see \haddockid{state22}).\par
\begin{quote}
{\haddockverb\begin{verbatim}
>>> let s1   = signal [Abst, Abst, Prst 1, Prst 2, Abst, Prst 3]

>>> hold 0 s1
{0,0,1,2,2,3}

\end{verbatim}}
\end{quote}(image: fig/moc-sy-pattern-hold.png)\par
           
\end{haddockdesc}
\begin{haddockdesc}
\item[\begin{tabular}{@{}l}
reactAbst2
\end{tabular}]\haddockbegindoc
\haddockbeginargs
\haddockdecltt{::} & \haddockdecltt{(Signal (AbstExt a1)
                                     -> Signal (AbstExt a2)
                                        -> Signal b)} & process which degrades the absent extension,
 e.g. holds present values \\
                                                        \haddockdecltt{->} & \haddockdecltt{Signal (AbstExt a1)} & \\
                                                                                                                   \haddockdecltt{->} & \haddockdecltt{Signal (AbstExt a2)} & \\
                                                                                                                                                                              \haddockdecltt{->} & \haddockdecltt{Signal (AbstExt b)} & absent-extended signal, properly reacting to the inputs \\
\end{tabulary}\par
Creates a wrapper enabling a reactive behvior to absent-extended
 signals for processes which would otherwise degrade the
 absent-extension (e.g. state machines with \haddocktt{ignore22} behavior).\par
Constructors: \haddocktt{reactAbst{\char 91}1-4{\char 93}}.\par
\begin{quote}
{\haddockverb\begin{verbatim}
>>> let s1 = readSignal "{1,1,1,_,1,_,1}" :: Signal (AbstExt Int)

>>> let proc = stated11 (B.ignore11 (+)) 0

>>> proc s1
{0,1,2,3,3,4,4,5}

>>> reactAbst1 proc s1
{0,1,2,⟂,3,⟂,4} 

\end{verbatim}}
\end{quote}(image: fig/moc-sy-pattern-reactAbst.png)\par
           
\end{haddockdesc}
\section{Interfaces}
\begin{haddockdesc}
\item[\begin{tabular}{@{}l}
toDE2
\end{tabular}]\haddockbegindoc
\haddockbeginargs
\haddockdecltt{::} & \haddockdecltt{Signal TimeStamp} & SY signal carrying \haddockid{DE} timestamps \\
                                                        \haddockdecltt{->} & \haddockdecltt{Signal a} & first input SY signal \\
                                                                                                        \haddockdecltt{->} & \haddockdecltt{Signal b} & second input SY signal \\
                                                                                                                                                        \haddockdecltt{->} & \haddockdecltt{(Signal a, Signal b)} & two output \haddockid{DE} signals \\
\end{tabulary}\par
Wraps explicit timestamps to a (set of) \haddockid{SY}
 signal(s), rendering the equivalent synchronized
 \haddockid{DE} signal(s).\par
Constructors: \haddocktt{toDE}, \haddocktt{toDE2}, \haddocktt{toDE3}, \haddocktt{toDE4}.\par
\begin{quote}
{\haddockverb\begin{verbatim}
>>> let s1 = SY.signal [0,3,4,6,9]

>>> let s2 = SY.signal [1,2,3,4,5]

>>> toDE s1 s2
{ 1 @0s, 2 @3s, 3 @4s, 4 @6s, 5 @9s}

\end{verbatim}}
\end{quote}(image: fig/moc-sy-tode.png)\par
           
\end{haddockdesc}
\begin{haddockdesc}
\item[\begin{tabular}{@{}l}
toSDF2\ ::\ Signal\ a\ ->\ Signal\ b\ ->\ (Signal\ a,\ Signal\ b)
\end{tabular}]\haddockbegindoc
Transforms a (set of) \haddockid{SY} signal(s) into the
 equivalent \haddockid{SDF} signal(s). The only change is
 the event consructor. The total order of SY is interpreted as
 partial order by the next SDF process downstream.\par
Constructors: \haddocktt{toSDF}, \haddocktt{toSDF2}, \haddocktt{toSDF3}, \haddocktt{toSDF4}.\par
\begin{quote}
{\haddockverb\begin{verbatim}
>>> let s = SY.signal [1,2,3,4,5]

>>> toSDF s
{1,2,3,4,5}

\end{verbatim}}
\end{quote}(image: fig/moc-sy-tosdf.png)\par
           
\end{haddockdesc}
\begin{haddockdesc}
\item[\begin{tabular}{@{}l}
zipx\ ::\ Vector\ (Signal\ a)\ ->\ Signal\ (Vector\ a)
\end{tabular}]\haddockbegindoc
Synchronizes all the signals contained by a vector and zips them
 into one signal of vectors. It instantiates the
 \haddockid{zipx} skeleton.\par
\begin{quote}
{\haddockverb\begin{verbatim}
>>> let s1 = SY.signal [1,2,3,4,5]

>>> let s2 = SY.signal [11,12,13,14,15]

>>> let v1 = V.vector [s1,s1,s2,s2]

>>> v1
<{1,2,3,4,5},{1,2,3,4,5},{11,12,13,14,15},{11,12,13,14,15}>

>>> zipx v1
{<1,1,11,11>,<2,2,12,12>,<3,3,13,13>,<4,4,14,14>,<5,5,15,15>}

\end{verbatim}}
\end{quote}(image: fig/moc-sy-zipx.png)\par
           
\end{haddockdesc}
\begin{haddockdesc}
\item[\begin{tabular}{@{}l}
unzipx\ ::\ Integer\ ->\ Signal\ (Vector\ a)\ ->\ Vector\ (Signal\ a)
\end{tabular}]\haddockbegindoc
Unzips the vectors carried by a signal into a vector of
 signals. It instantiates the \haddockid{unzipx}
 skeleton. To avoid infinite recurrence, the user needs to provide
 the length of the output vector.\par
\begin{quote}
{\haddockverb\begin{verbatim}
>>> let v1 = V.vector [1,2,3,4]

>>> let s1 = SY.signal [v1,v1,v1,v1,v1]

>>> s1
{<1,2,3,4>,<1,2,3,4>,<1,2,3,4>,<1,2,3,4>,<1,2,3,4>}

>>> unzipx 4 s1
<{1,1,1,1,1},{2,2,2,2,2},{3,3,3,3,3},{4,4,4,4,4}>

\end{verbatim}}
\end{quote}(image: fig/moc-sy-zipx.png)\par
           
\end{haddockdesc}
\begin{haddockdesc}
\item[\begin{tabular}{@{}l}
unzipx'\ ::\ Signal\ (Vector\ a)\ ->\ Vector\ (Signal\ a)
\end{tabular}]\haddockbegindoc
Same as \haddockid{unzipx}, but "sniffs" the first event to determine the length of the output vector. Might have unsafe behavior!\par
\begin{quote}
{\haddockverb\begin{verbatim}
>>> let v1 = V.vector [1,2,3,4]

>>> let s1 = SY.signal [v1,v1,v1,v1,v1]

>>> s1
{<1,2,3,4>,<1,2,3,4>,<1,2,3,4>,<1,2,3,4>,<1,2,3,4>}

>>> unzipx' s1
<{1,1,1,1,1},{2,2,2,2,2},{3,3,3,3,3},{4,4,4,4,4}>

\end{verbatim}}
\end{quote}
\end{haddockdesc}