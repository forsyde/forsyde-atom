\haddockmoduleheading{ForSyDe.Atom.Utility.Tuple}
\label{module:ForSyDe.Atom.Utility.Tuple}
\haddockbeginheader
{\haddockverb\begin{verbatim}
module ForSyDe.Atom.Utility.Tuple (
    at22,  (||<),  (<>),  ($$)
  ) where\end{verbatim}}
\haddockendheader

This module implements general purpose utility functions. It mainly
 hosts functions dealing with tuples. Utility are provided for up
 until 9-tuples. Follow the examples in the source code in case it
 does not suffice.\par
\textbf{IMPORTANT!!!} Most of the multi-parameter higher-order functions
 provided by the library API are named along the lines of
 \haddocktt{functionMN} where \haddocktt{M} represents the number of \textbf{\emph{curried}}
 inputs (i.e. \haddocktt{a1\ ->\ a2\ ->\ ...\ ->\ aM}), while \haddocktt{N} represents the
 number of \textbf{\emph{tupled}} outputs (i.e. \haddocktt{(b1,b2,...,bN)}). To avoid
 repetition we shall only provide documentation for functions with 2
 inputs and 2 outputs (i.e. \haddocktt{function22}).\par

\begin{haddockdesc}
\item[\begin{tabular}{@{}l}
at22\ ::\ (a1,\ a2)\ ->\ a2
\end{tabular}]\haddockbegindoc
The \haddocktt{at}\emph{xy} functions return the \emph{y}-th element of an \emph{x}-tuple.\par
\haddocktt{ForSyDe.Atom.Utility} exports the constructors below. Please
 follow the examples in the source code if they do not suffice:\par
\begin{quote}
{\haddockverb\begin{verbatim}
at21, at22,
at31, at32, at33,
at41, at42, at43, at44,
at51, at52, at53, at54, at55,
at61, at62, at63, at64, at65, at66, 
at71, at72, at73, at74, at75, at76, at77,
at81, at82, at83, at84, at85, at86, at87, at88,
at91, at92, at93, at94, at95, at96, at97, at98, at99,\end{verbatim}}
\end{quote}
Example:\par
\begin{quote}
{\haddockverb\begin{verbatim}
>>> at53 (1,2,3,4,5)
3

\end{verbatim}}
\end{quote}
\end{haddockdesc}
\begin{haddockdesc}
\item[\begin{tabular}{@{}l}
(||<)\ ::\ (Functor\ f1,\ Functor\ f2)\ =>\\\ \ \ \ \ \ \ \ \ f1\ (f2\ (a1,\ a2))\ ->\ (f1\ (f2\ a1),\ f1\ (f2\ a2))
\end{tabular}]\haddockbegindoc
This set of utility functions "unzip" nested n-tuples, provided
 as postfix operators. They are crucial for reconstructing data
 types from higher-order functions which input functions with
 multiple outputs. It relies on the nested types being instances of
 \haddockid{Functor}.\par
The operator convention is \haddocktt{(|+<+)}, where the number of \haddocktt{|}
 represent the number of layers the n-tuple is lifted, while the
 number of \haddocktt{<} + 1 is the order \emph{n} of the n-tuple.\par
\haddocktt{ForSyDe.Atom.Utility} exports the constructors below. Please
 follow the examples in the source code if they do not suffice:\par
\begin{quote}
{\haddockverb\begin{verbatim}
   |<,    |<<,    |<<<,    |<<<<,    |<<<<<,    |<<<<<<,    |<<<<<<<,    |<<<<<<<<,
  ||<,   ||<<,   ||<<<,   ||<<<<,   ||<<<<<,   ||<<<<<<,   ||<<<<<<<,   ||<<<<<<<<,
 |||<,  |||<<,  |||<<<,  |||<<<<,  |||<<<<<,  |||<<<<<<,  |||<<<<<<<,  |||<<<<<<<<,  
||||<, ||||<<, ||||<<<, ||||<<<<, ||||<<<<<, ||||<<<<<<, ||||<<<<<<<, ||||<<<<<<<<, \end{verbatim}}
\end{quote}
Example:\par
\begin{quote}
{\haddockverb\begin{verbatim}
>>> :set -XPostfixOperators

>>> ([Just (1,2,3), Nothing, Just (4,5,6)] ||<<)
([Just 1,Nothing,Just 4],[Just 2,Nothing,Just 5],[Just 3,Nothing,Just 6])

\end{verbatim}}
\end{quote}
\end{haddockdesc}
\begin{haddockdesc}
\item[\begin{tabular}{@{}l}
(<>)\ ::\ (a1\ ->\ a2\ ->\ b1)\ ->\ (a1,\ a2)\ ->\ b1
\end{tabular}]\haddockbegindoc
Infix currying operators used for convenience. \par
The operator convention is \haddocktt{(<>+)}, where the number of \haddocktt{>} + 1 is
 the order \emph{n} of the n-tuple.\par
\haddocktt{ForSyDe.Atom.Utility} exports the constructors below. Please
 follow the examples in the source code if they do not suffice:\par
\begin{quote}
{\haddockverb\begin{verbatim}
<>, <>>, <>>>, <>>>>, <>>>>>, <>>>>>>, <>>>>>>>, <>>>>>>>>\end{verbatim}}
\end{quote}
Example:\par
\begin{quote}
{\haddockverb\begin{verbatim}
>>> (+) <> (1,2)
3

\end{verbatim}}
\end{quote}
\end{haddockdesc}
\begin{haddockdesc}
\item[\begin{tabular}{@{}l}
({\char '44}{\char '44})\ ::\ (a1\ ->\ b1,\ a2\ ->\ b2)\ ->\ (a1,\ a2)\ ->\ (b1,\ b2)
\end{tabular}]\haddockbegindoc
Infix function application operator for tuples. \par
The operator convention is \haddocktt{({\char '44}+)}, where the number of \haddocktt{{\char '44}} is the
 order \emph{n} of the n-tuple. For Applying a function on nontuples we
 rely on \haddockid{{\char '44}} provided by \haddocktt{Prelude}.\par
\haddocktt{ForSyDe.Atom.Utility} exports the constructors below. Please
 follow the examples in the source code if they do not suffice:\par
\begin{quote}
{\haddockverb\begin{verbatim}
$$, $$$, $$$$, $$$$$, $$$$$$, $$$$$$$, $$$$$$$$, $$$$$$$$$\end{verbatim}}
\end{quote}
Example:\par
\begin{quote}
{\haddockverb\begin{verbatim}
>>> ((+),(-)) $$ (1,1) $$ (2,2)
(3,-1)

\end{verbatim}}
\end{quote}
\end{haddockdesc}