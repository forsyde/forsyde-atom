\haddockmoduleheading{ForSyDe.Atom.MoC.SDF}
\label{module:ForSyDe.Atom.MoC.SDF}
\haddockbeginheader
{\haddockverb\begin{verbatim}
module ForSyDe.Atom.MoC.SDF (
    SDF(SDF, val),  Signal,  Prod,  Cons,  signal,  readSignal,  delay, 
    delay',  comb22,  reconfig22,  constant2,  generate2,  stated22,  state22, 
    moore22,  mealy22,  toSY2,  zipx,  unzipx
  ) where\end{verbatim}}
\haddockendheader

The \haddocktt{SDF} library implements the atoms holding the sematics for the
 synchronous data flow computation model. It also provides a set of
 helpers for properly instantiating process network patterns as
 process constructors.\par
\textbf{IMPORTANT!!!}
 see the \href{ForSyDe-Atom.html#naming_conv}{naming convention} rules
 on how to interpret, use and develop your own constructors.\par

\section{Synchronous data flow (\haddocktt{SDF}) event}
The synchronous data flow (\haddocktt{SDF}) MoC is the first untimed MoC
 implemented by the \haddocktt{forsyde-atom} framework. On untimed MoCs,
 \href{ForSyDe-Atom.html#lee98}{[Lee98]} states that: "when tags are
 partially ordered rather than totally ordered, we say that the
 system is untimed. Untimed systems cannot have the same notion of
 causality as timed systems {\char 91}see \haddockid{SY}{\char 93}. (...)
 Processes defined in terms of constraints on the tags in the
 signals (...) have a \emph{consistent cut} rather than
 \emph{simultaneity}."  Regarding SDF, it states that "is a special
 case of Kahn process networks
 \href{ForSyDe-Atom.html#kahn76}{[Kahn76]}. A dataflow process is a Kahn
 process that is also sequential, where the events on the
 self-loop signal denote the firings of the dataflow actor. The
 firing rules of a dataflow actor are partial ordering constraints
 between these events and events on the inputs. (...)
 Produced/consumed events are defined in terms of relations with
 the events in the firing signal. It results that for the same
 firing \emph{i}, eᵢ < eₒ, as an intuitive sort of
 causality constraint."\par
Based on the above insights, we can formulate a simplified
 definition of the \haddocktt{forsyde-atom} interpretation of SDF:\par
\begin{description}
\item[The SDF MoC] is abstracting the execution semantics of a system
 where computation is performed according to firing rules where
 the production and the consumption rates are fixed.
\end{description}Below is a \emph{possible} behavior in time of the input and the
 output signals of a SDF process. Events sharing the same partial
 ordering in relation to one firing are overlined:\par
                 (image: fig/moc-sdf-example.png)\par
                 Implementing the SDF tag system implied a series of engineering
 decisions which lead to the following particularities:\par
                 \begin{enumerate}
                 \item 
                 signals represent FIFO channels, and tags are implicit from
 their position in the \haddockid{Stream}
 structure. Internally, \haddockid{SDF} signals have
 exactly the same structure as \haddockid{SY} signals,
 whereas the partial ordering is imposed by the processes alone.\par
                 
                 \item 
                 the \haddockid{SDF} event constructor wraps only a
 value.\par
                 
                 \item 
                 being an \emph{untimed MoC}, the order between events is partial to
 the firings of processes. An SDF atom will fire only when there
 are enough events to trigger its inputs. Once a firing occurs, it
 will take care of partitioning the input or output signals.\par
                 
                 \item 
                 SDF atoms \emph{do} require a context: the consumption \emph{c} and
 production \emph{p} rates. Also, the functions passed as arguments
 reflect the fact that multiple events are handled during a
 firing.\par
                 
                 \item 
                 the previous statement can be synthesized into the following
 \href{ForSyDe-Atom-MoC.html#context}{execution context}, which also
 justifies the SDF implementation of \haddockid{Fun} and
 for \haddockid{Ret}:\par
                 
                 \end{enumerate}
                 (image: fig/eqs-moc-sdf-context.png)\par
                 
\begin{haddockdesc}
\item[\begin{tabular}{@{}l}
newtype\ SDF\ a
\end{tabular}]\haddockbegindoc
\haddockbeginconstrs
\haddockdecltt{=} & \haddockdecltt{SDF} & \\
                    \haddockdecltt{val :: a} &
\end{tabulary}\par
\haddockbeginconstrs
\end{tabulary}\par
The CT type, identifying a discrete time event and implementing an
 instance of the \haddockid{MoC} class. A discrete event explicitates its tag
 which is represented as an integer.\par

\end{haddockdesc}
\begin{haddockdesc}
\item[\begin{tabular}{@{}l}
instance\ Functor\ SDF
\end{tabular}]\haddockbegindoc
Allows for mapping of functions on a SDF event.\par

\end{haddockdesc}
\begin{haddockdesc}
\item[\begin{tabular}{@{}l}
instance\ Applicative\ SDF
\end{tabular}]\haddockbegindoc
Allows for lifting functions on a pair of SDF events.\par

\end{haddockdesc}
\begin{haddockdesc}
\item[\begin{tabular}{@{}l}
instance\ Foldable\ SDF\\instance\ Traversable\ SDF
\end{tabular}]
\end{haddockdesc}
\begin{haddockdesc}
\item[\begin{tabular}{@{}l}
instance\ MoC\ SDF
\end{tabular}]\haddockbegindoc
Implenents the SDF semantics for the MoC atoms\par

\end{haddockdesc}
\begin{haddockdesc}
\item[\begin{tabular}{@{}l}
instance\ Read\ a\ =>\ Read\ (SDF\ a)
\end{tabular}]\haddockbegindoc
Reads the value wrapped\par

\end{haddockdesc}
\begin{haddockdesc}
\item[\begin{tabular}{@{}l}
instance\ Show\ a\ =>\ Show\ (SDF\ a)
\end{tabular}]\haddockbegindoc
Shows the value wrapped\par

\end{haddockdesc}
\begin{haddockdesc}
\item[\begin{tabular}{@{}l}
instance\ Plottable\ a\ =>\ Plot\ (Signal\ a)
\end{tabular}]\haddockbegindoc
\haddockid{SDF} signals.\par

\end{haddockdesc}
\begin{haddockdesc}
\item[\begin{tabular}{@{}l}
instance\ type\ Ret\ SDF\ a\ =\ (Prod,\ {\char 91}a{\char 93})\\instance\ type\ Fun\ SDF\ a\ b\ =\ (Cons,\ {\char 91}a{\char 93}\ ->\ b)
\end{tabular}]
\end{haddockdesc}
\section{Aliases {\char '46} utilities}
A set of type synonyms and utilities are provided for
 convenience. The API type signatures will feature these aliases
 to hide the cumbersome construction of atoms and patters as seen
 in \haddocktt{ForSyDe.Atom.MoC}.\par

\begin{haddockdesc}
\item[\begin{tabular}{@{}l}
type\ Signal\ a\ =\ Stream\ (SDF\ a)
\end{tabular}]\haddockbegindoc
Type synonym for a SY signal, i.e. "a signal of SY events"\par

\end{haddockdesc}
\begin{haddockdesc}
\item[\begin{tabular}{@{}l}
type\ Prod\ =\ Int
\end{tabular}]\haddockbegindoc
Type synonym for consumption rate\par

\end{haddockdesc}
\begin{haddockdesc}
\item[\begin{tabular}{@{}l}
type\ Cons\ =\ Int
\end{tabular}]\haddockbegindoc
Type synonym for production rate\par

\end{haddockdesc}
\begin{haddockdesc}
\item[\begin{tabular}{@{}l}
signal\ ::\ {\char 91}a{\char 93}\ ->\ Signal\ a
\end{tabular}]\haddockbegindoc
Transforms a list of values into a SDF signal with only one
 partition, i.e. all events share the same (initial) tag.\par

\end{haddockdesc}
\begin{haddockdesc}
\item[\begin{tabular}{@{}l}
readSignal\ ::\ Read\ a\ =>\ String\ ->\ Signal\ a
\end{tabular}]\haddockbegindoc
Reads a signal from a string. Like with the \haddocktt{read} function from
 \haddocktt{Prelude}, you must specify the tipe of the signal.\par
\begin{quote}
{\haddockverb\begin{verbatim}
>>> readSignal "{1,2,3,4,5}" :: Signal Int
{1,2,3,4,5}

\end{verbatim}}
\end{quote}
\end{haddockdesc}
These SY process constructors are basically specific
 instantiations of the patterns of atoms defined in
 \haddocktt{ForSyDe.Atom.MoC}. Some are also wrapping functions in an
 extended behavioural model.\par

\subsection{Simple}
\begin{haddockdesc}
\item[\begin{tabular}{@{}l}
delay
\end{tabular}]\haddockbegindoc
\haddockbeginargs
\haddockdecltt{::} & \haddockdecltt{[a]} & list of initial values \\
                                           \haddockdecltt{->} & \haddockdecltt{Signal a} & input signal \\
                                                                                           \haddockdecltt{->} & \haddockdecltt{Signal a} & output signal \\
\end{tabulary}\par
The \haddocktt{delay} process "delays" a signal with initial events built
 from a list. It is an instantiation of the \haddockid{delay}
 constructor.\par
\begin{quote}
{\haddockverb\begin{verbatim}
>>> let s = signal [1,2,3,4,5]

>>> delay [0,0,0] s
{0,0,0,1,2,3,4,5}

\end{verbatim}}
\end{quote}(image: fig/moc-sdf-pattern-delay.png)\par
           
\end{haddockdesc}
\begin{haddockdesc}
\item[\begin{tabular}{@{}l}
delay'
\end{tabular}]\haddockbegindoc
\haddockbeginargs
\haddockdecltt{::} & \haddockdecltt{Signal a} & signal containing the initial tokens \\
                                                \haddockdecltt{->} & \haddockdecltt{Signal a} & input signal \\
                                                                                                \haddockdecltt{->} & \haddockdecltt{Signal a} & output signal \\
\end{tabulary}\par
Similar to the previous, but this is the raw instantiation of the
 \haddockid{delay} pattern. It appends the contents of one
 signal at the head of another signal.\par
\begin{quote}
{\haddockverb\begin{verbatim}
>>> let s1 = signal [0,0,0]

>>> let s2 = signal [1,2,3,4,5]

>>> delay' s1 s2
{0,0,0,1,2,3,4,5}

\end{verbatim}}
\end{quote}(image: fig/moc-sdf-pattern-delayp.png)\par
           
\end{haddockdesc}
\begin{haddockdesc}
\item[\begin{tabular}{@{}l}
comb22
\end{tabular}]\haddockbegindoc
\haddockbeginargs
\haddockdecltt{::} & \haddockdecltt{((Cons, Cons), (Prod, Prod), [a1]
                                                                 -> [a2]
                                                                    -> ([b1], [b2]))} & function on lists of values, tupled with consumption /
 production rates \\
                                                                                        \haddockdecltt{->} & \haddockdecltt{Signal a1} & first input signal \\
                                                                                                                                         \haddockdecltt{->} & \haddockdecltt{Signal a2} & second input signal \\
                                                                                                                                                                                          \haddockdecltt{->} & \haddockdecltt{(Signal b1, Signal b2)} & two output signals \\
\end{tabulary}\par
\haddocktt{comb} processes map combinatorial functions on signals and take
 care of synchronization between input signals. It instantiates the
 \haddocktt{comb} atom pattern (see \haddockid{comb22}).\par
Constructors: \haddocktt{comb{\char 91}1-4{\char 93}{\char 91}1-4{\char 93}}.\par
\begin{quote}
{\haddockverb\begin{verbatim}
>>> let s1 = signal [1..]

>>> let s2 = signal [1,1,1,1,1,1,1]

>>> let f [a,b,c] [d,e] = [a+d, c+e]

>>> comb21 ((3,2),2,f) s1 s2
{2,4,5,7,8,10}

\end{verbatim}}
\end{quote}Incorrect usage (not covered by \haddocktt{doctest}):\par
           \begin{quote}
           {\haddockverb\begin{verbatim}
           λ> comb21 ((3,2),3,f) s1 s2
*** Exception: [MoC.SDF] Wrong production\end{verbatim}}
           \end{quote}
           (image: fig/moc-sdf-pattern-comb.png)\par
           
\end{haddockdesc}
\begin{haddockdesc}
\item[\begin{tabular}{@{}l}
reconfig22
\end{tabular}]\haddockbegindoc
\haddockbeginargs
\haddockdecltt{::} & \haddockdecltt{((Cons, Cons), (Prod, Prod))} & \\
                                                                    \haddockdecltt{->} & \haddockdecltt{Signal ([a1]
                                                                                                                -> [a2]
                                                                                                                   -> ([b1], [b2]))} & function on lists of values, tupled with consumption /
 production rates \\
                                                                                                                                       \haddockdecltt{->} & \haddockdecltt{Signal a1} & first input signal \\
                                                                                                                                                                                        \haddockdecltt{->} & \haddockdecltt{Signal a2} & second input signal \\
                                                                                                                                                                                                                                         \haddockdecltt{->} & \haddockdecltt{(Signal b1, Signal b2)} & two output signals \\
\end{tabulary}\par
\haddocktt{reconfig} creates an SDF adaptive process where the first signal
 carries functions and the other carry the arguments. It
 instantiates the \haddocktt{reconfig} atom pattern (see
 \haddockid{reconfig22}). According to our SDF definition,
 the production and consumption rates need to be fixed, so they are
 passed as parameters to the constructor, whereas the first signal
 carries adaptive functions only. For the adaptive signal it only
 makes sense that the consumption rate is always 1.\par
Constructors: \haddocktt{reconfig{\char 91}1-4{\char 93}{\char 91}1-4{\char 93}}.\par
\begin{quote}
{\haddockverb\begin{verbatim}
>>> let f1 a = [sum a]

>>> let f2 a = [maximum a]

>>> let sf = signal [f1,f2,f1,f2,f1,f2,f1]

>>> let s1 = signal [1..]

>>> reconfig11 (4,1) sf s1
{10,8,42,16,74,24,106}

\end{verbatim}}
\end{quote}(image: fig/moc-sdf-pattern-reconfig.png)\par
           
\end{haddockdesc}
\begin{haddockdesc}
\item[\begin{tabular}{@{}l}
constant2
\end{tabular}]\haddockbegindoc
\haddockbeginargs
\haddockdecltt{::} & \haddockdecltt{([b1], [b2])} & values to be repeated \\
                                                    \haddockdecltt{->} & \haddockdecltt{(Signal b1, Signal b2)} & generated signals \\
\end{tabulary}\par
A signal generator which repeats the initial tokens
 indefinitely. It is actually an instantiation of the \haddocktt{stated0X}
 constructor (check \haddockid{stated22}).\par
Constructors: \haddocktt{constant{\char 91}1-4{\char 93}}.\par
\begin{quote}
{\haddockverb\begin{verbatim}
>>> let (s1, s2) = constant2 ([1,2,3],[2,1])

>>> takeS 7 s1
{1,2,3,1,2,3,1}

>>> takeS 5 s2
{2,1,2,1,2}

\end{verbatim}}
\end{quote}(image: fig/moc-sdf-pattern-constant.png)\par
           
\end{haddockdesc}
\begin{haddockdesc}
\item[\begin{tabular}{@{}l}
generate2
\end{tabular}]\haddockbegindoc
\haddockbeginargs
\haddockdecltt{::} & \haddockdecltt{((Cons, Cons), (Prod, Prod), [b1]
                                                                 -> [b2]
                                                                    -> ([b1], [b2]))} & function to generate next value, tupled with
 consumption / production rates \\
                                                                                        \haddockdecltt{->} & \haddockdecltt{([b1], [b2])} & values of initial tokens \\
                                                                                                                                            \haddockdecltt{->} & \haddockdecltt{(Signal b1, Signal b2)} & generated signals \\
\end{tabulary}\par
A signal generator based on a function and a kernel value. It
 is actually an instantiation of the \haddocktt{stated0X} constructor
 (check \haddockid{stated22}).\par
Constructors: \haddocktt{generate{\char 91}1-4{\char 93}}.\par
\begin{quote}
{\haddockverb\begin{verbatim}
>>> let f a b = ([sum a, sum a],[sum b, sum b, sum b])

>>> let (s1,s2) = generate2 ((2,3),(2,3),f) ([1,1],[2,2,2])

>>> takeS 7 s1
{1,1,2,2,4,4,8}

>>> takeS 8 s2
{2,2,2,6,6,6,18,18}

\end{verbatim}}
\end{quote}(image: fig/moc-sdf-pattern-generate.png)\par
           
\end{haddockdesc}
\begin{haddockdesc}
\item[\begin{tabular}{@{}l}
stated22
\end{tabular}]\haddockbegindoc
\haddockbeginargs
\haddockdecltt{::} & \haddockdecltt{((Cons, Cons, Cons, Cons), (Prod, Prod), [b1]
                                                                             -> [b2]
                                                                                -> [a1]
                                                                                   -> [a2]
                                                                                      -> ([b1], [b2]))} & next state function, tupled with
 consumption / production rates \\
                                                                                                          \haddockdecltt{->} & \haddockdecltt{([b1], [b2])} & initial state partitions of values \\
                                                                                                                                                              \haddockdecltt{->} & \haddockdecltt{Signal a1} & first input signal \\
                                                                                                                                                                                                               \haddockdecltt{->} & \haddockdecltt{Signal a2} & second input signal \\
                                                                                                                                                                                                                                                                \haddockdecltt{->} & \haddockdecltt{(Signal b1, Signal b2)} & output signals \\
\end{tabulary}\par
\haddocktt{stated} is a state machine without an output decoder. It is an
 instantiation of the \haddocktt{state} MoC constructor (see
 \haddockid{stated22}).\par
Constructors: \haddocktt{stated{\char 91}1-4{\char 93}{\char 91}1-4{\char 93}}.\par
\begin{quote}
{\haddockverb\begin{verbatim}
>>> let f [a] [b,c] = [a+b+c]

>>> let s = signal [1,2,3,4,5,6,7]

>>> stated11 ((1,2),1,f) [1] s
{1,4,11,22}

\end{verbatim}}
\end{quote}(image: fig/moc-sdf-pattern-stated.png)\par
           
\end{haddockdesc}
\begin{haddockdesc}
\item[\begin{tabular}{@{}l}
state22
\end{tabular}]\haddockbegindoc
\haddockbeginargs
\haddockdecltt{::} & \haddockdecltt{((Cons, Cons, Cons, Cons), (Prod, Prod), [b1]
                                                                             -> [b2]
                                                                                -> [a1]
                                                                                   -> [a2]
                                                                                      -> ([b1], [b2]))} & next state function, tupled with consumption /
 production rates \\
                                                                                                          \haddockdecltt{->} & \haddockdecltt{([b1], [b2])} & initial partitions of values \\
                                                                                                                                                              \haddockdecltt{->} & \haddockdecltt{Signal a1} & first input signal \\
                                                                                                                                                                                                               \haddockdecltt{->} & \haddockdecltt{Signal a2} & second input signal \\
                                                                                                                                                                                                                                                                \haddockdecltt{->} & \haddockdecltt{(Signal b1, Signal b2)} & output signals \\
\end{tabulary}\par
\haddocktt{state} is a state machine without an output decoder. It is an
 instantiation of the \haddocktt{stated} MoC constructor (see
 \haddockid{state22}).\par
Constructors: \haddocktt{state{\char 91}1-4{\char 93}{\char 91}1-4{\char 93}}.\par
\begin{quote}
{\haddockverb\begin{verbatim}
>>> let f [a] [b,c] = [a+b+c]

>>> let s = signal [1,2,3,4,5,6,7]

>>> state11 ((1,2),1,f) [1] s
{4,11,22}

\end{verbatim}}
\end{quote}(image: fig/moc-sdf-pattern-state.png)\par
           
\end{haddockdesc}
\begin{haddockdesc}
\item[\begin{tabular}{@{}l}
moore22
\end{tabular}]\haddockbegindoc
\haddockbeginargs
\haddockdecltt{::} & \haddockdecltt{((Cons, Cons, Cons), Prod, [st]
                                                               -> [a1]
                                                                  -> [a2]
                                                                     -> [st])} & next state function, tupled with consumption / production
 rates \\
                                                                                 \haddockdecltt{->} & \haddockdecltt{(Cons, (Prod, Prod), [st]
                                                                                                                                          -> ([b1], [b2]))} & output decoder, tupled with consumption / production
 rates \\
                                                                                                                                                              \haddockdecltt{->} & \haddockdecltt{[st]} & initial state values \\
                                                                                                                                                                                                          \haddockdecltt{->} & \haddockdecltt{Signal a1} & \\
                                                                                                                                                                                                                                                           \haddockdecltt{->} & \haddockdecltt{Signal a2} & \\
                                                                                                                                                                                                                                                                                                            \haddockdecltt{->} & \haddockdecltt{(Signal b1, Signal b2)} & \\
\end{tabulary}\par
\haddocktt{moore} processes model Moore state machines. It is an
 instantiation of the \haddocktt{moore} MoC constructor (see
 \haddockid{moore22}).\par
Constructors: \haddocktt{moore{\char 91}1-4{\char 93}{\char 91}1-4{\char 93}}.\par
\begin{quote}
{\haddockverb\begin{verbatim}
>>> let ns [a] [b,c] = [a+b+c]

>>> let od [a]       = [a+1,a*2]

>>> let s = signal [1,2,3,4,5,6,7]

>>> moore11 ((1,2),1,ns) (1,2,od) [1] s
{2,2,5,8,12,22,23,44}

\end{verbatim}}
\end{quote}(image: fig/moc-sdf-pattern-moore.png)\par
           
\end{haddockdesc}
\begin{haddockdesc}
\item[\begin{tabular}{@{}l}
mealy22
\end{tabular}]\haddockbegindoc
\haddockbeginargs
\haddockdecltt{::} & \haddockdecltt{((Cons, Cons, Cons), Prod, [st]
                                                               -> [a1]
                                                                  -> [a2]
                                                                     -> [st])} & next state function, tupled with consumption / production
 rates \\
                                                                                 \haddockdecltt{->} & \haddockdecltt{((Cons, Cons, Cons), (Prod, Prod), [st]
                                                                                                                                                        -> [a1]
                                                                                                                                                           -> [a2]
                                                                                                                                                              -> ([b1], [b2]))} & outpt decoder, tupled with consumption / production rates \\
                                                                                                                                                                                  \haddockdecltt{->} & \haddockdecltt{[st]} & initial state values \\
                                                                                                                                                                                                                              \haddockdecltt{->} & \haddockdecltt{Signal a1} & \\
                                                                                                                                                                                                                                                                               \haddockdecltt{->} & \haddockdecltt{Signal a2} & \\
                                                                                                                                                                                                                                                                                                                                \haddockdecltt{->} & \haddockdecltt{(Signal b1, Signal b2)} & \\
\end{tabulary}\par
\haddocktt{mealy} processes model Mealy state machines. It is an
 instantiation of the \haddocktt{mealy} MoC constructor
 (see \haddockid{mealy22}).\par
Constructors: \haddocktt{mealy{\char 91}1-4{\char 93}{\char 91}1-4{\char 93}}.\par
\begin{quote}
{\haddockverb\begin{verbatim}
>>> let ns [a] [b,c] = [a+b+c]

>>> let od [a] [b]   = [a+b,a*b]

>>> let s = signal [1,2,3,4,5,6,7]

>>> mealy11 ((1,2),1,ns) ((1,1),2,od) [1] s
{2,1,6,8,14,33,26,88}

\end{verbatim}}
\end{quote}(image: fig/moc-sdf-pattern-mealy.png)\par
           
\end{haddockdesc}
\subsection{Interfaces}
\begin{haddockdesc}
\item[\begin{tabular}{@{}l}
toSY2\ ::\ Signal\ a\ ->\ Signal\ b\ ->\ (Signal\ a,\ Signal\ b)
\end{tabular}]\haddockbegindoc
Transforms a (set of) \haddockid{SDF} signal(s) into
 the equivalent \haddockid{SY} signal(s). The only change
 is the event consructor. The partial order of DE is interpreted as
 SY's total order, based on the positioning of events in the signals
 (e.g. FIFO buffers) at that moment.\par
Constructors: \haddocktt{toSY{\char 91}1-4{\char 93}}.\par
\begin{quote}
{\haddockverb\begin{verbatim}
>>> let s = SDF.signal [1,2,3,4,5]

>>> toSY s
{1,2,3,4,5}

\end{verbatim}}
\end{quote}(image: fig/moc-sdf-tosy.png)\par
           
\end{haddockdesc}
\begin{haddockdesc}
\item[\begin{tabular}{@{}l}
zipx
\end{tabular}]\haddockbegindoc
\haddockbeginargs
\haddockdecltt{::} & \haddockdecltt{Vector Cons} & consumption rates \\
                                                   \haddockdecltt{->} & \haddockdecltt{Vector (Signal a)} & vector of signals \\
                                                                                                            \haddockdecltt{->} & \haddockdecltt{Signal (Vector a)} & signal of vectors \\
\end{tabulary}\par
Consumes tokens from a vector of signals and merges them into a
 signal of vectors, with a production rate of 1. It instantiates the
 \haddockid{zipx} skeleton.\par
\begin{quote}
{\haddockverb\begin{verbatim}
>>> let s1 = SDF.signal [1,2,3,4,5]

>>> let s2 = SDF.signal [11,12,13,14,15]

>>> let v1 = V.vector [s1,s1,s2,s2]

>>> let r  = V.vector [2,1,2,1]

>>> v1
<{1,2,3,4,5},{1,2,3,4,5},{11,12,13,14,15},{11,12,13,14,15}>

>>> zipx r v1
{<1,2,1,11,12,11>,<3,4,2,13,14,12>}

\end{verbatim}}
\end{quote}(image: fig/moc-sdf-zipx.png)\par
           
\end{haddockdesc}
\begin{haddockdesc}
\item[\begin{tabular}{@{}l}
unzipx
\end{tabular}]\haddockbegindoc
\haddockbeginargs
\haddockdecltt{::} & \haddockdecltt{Vector Prod} & production rates (in reverse order) \\
                                                   \haddockdecltt{->} & \haddockdecltt{Signal (Vector a)} & signal of vectors \\
                                                                                                            \haddockdecltt{->} & \haddockdecltt{Vector (Signal a)} & vector of signals \\
\end{tabulary}\par
Consumes the vectors carried by a signal with a rate of 1, and
 unzips them into a vector of signals based on the user provided
 rates. It instantiates the \haddockid{unzipx}
 skeleton.\par
\textbf{OBS:} due to the \haddockid{recur} pattern
 contained by \haddockid{unzipx}, the vector of
 production rates needs to be provided in reverse order (see
 \haddocktt{ForSyDe.Atom.Skeleton.Vector}).\par
\begin{quote}
{\haddockverb\begin{verbatim}
>>> let s1 = SDF.signal [1,2,3,4,5]

>>> let s2 = SDF.signal [11,12,13,14,15]

>>> let v1 = V.vector [s1,s1,s2,s2]

>>> let r  = V.vector [2,1,2,1]

>>> let sz = zipx r v1

>>> v1
<{1,2,3,4,5},{1,2,3,4,5},{11,12,13,14,15},{11,12,13,14,15}>

>>> sz
{<1,2,1,11,12,11>,<3,4,2,13,14,12>}

>>> unzipx (V.reverse r) sz
<{1,2,3,4},{1,2},{11,12,13,14},{11,12}>

\end{verbatim}}
\end{quote}(image: fig/moc-sdf-unzipx.png)\par
           
\end{haddockdesc}