\haddockmoduleheading{ForSyDe.Atom.ExB}
\label{module:ForSyDe.Atom.ExB}
\haddockbeginheader
{\haddockverb\begin{verbatim}
module ForSyDe.Atom.ExB (
    ExB(extend, (/.\), (/*\), (/&\), (/!\)),  res22,  filter,  filter', 
    degrade,  ignore22
  ) where\end{verbatim}}
\haddockendheader

This module exports the core entities of the extended behavior
 layer: interfaces for atoms and common patterns of atoms. It does
 \emph{NOT} export any implementation or instantiation of any specific
 behavior extension type.\par
\textbf{IMPORTANT!!!}
 see the \href{ForSyDe-Atom.html#naming_conv}{naming convention} rules
 on how to interpret, use and develop your own constructors.\par

\section{Atoms}
\begin{haddockdesc}
\item[\begin{tabular}{@{}l}
class\ Functor\ b\ =>\ ExB\ b\ where
\end{tabular}]\haddockbegindoc
Class which defines the atoms for the extended behavior layer.\par
As its name suggests, this layer is extending the behavior of
 processes (or merely of functions if we ignore timing semantics
 completely), by expanding the domains of the wrapped layer
 (e.g. the set of values) with symbols having clearly-defined
 semantics (e.g. special events with known responses).\par
The types associated with this layer can simply be describes as:\par
(image: fig/eqs-exb-types.png)\par
where  \emph{α} is a base type and \emph{b} is the type extension,
 i.e. a set of symbols with clearly-defined semantics.\par
Extended behavior atoms are functions of these types, defined as
 interfaces in the \haddockid{ExB} type class.\par

\haddockpremethods{}\textbf{Methods}
\end{haddockdesc}
\begin{haddockdesc}
\item[\begin{tabular}{@{}l}
instance\ ExB\ AbstExt
\end{tabular}]\haddockbegindoc
Implements the absent semantics of the extended behavior atoms.\par

\end{haddockdesc}
\section{Patterns}
\begin{haddockdesc}
\item[\begin{tabular}{@{}l}
res22
\end{tabular}]\haddockbegindoc
\haddockbeginargs
\haddockdecltt{::} & ExB b \\
                     \haddockdecltt{=>} & \haddockdecltt{(a1
                                                          -> a2
                                                             -> (a1', a2'))} & function on values \\
                                                                               \haddockdecltt{->} & \haddockdecltt{b a1} & first input \\
                                                                                                                           \haddockdecltt{->} & \haddockdecltt{b a2} & second input \\
                                                                                                                                                                       \haddockdecltt{->} & \haddockdecltt{(b a1', b a2')} & tupled output \\
\end{tabulary}\par
(image: fig/eqs-exb-pattern-resolution.png)\par
The \haddocktt{res} behavior pattern lifts a function on values to the
 extended behavior domain, and applies a resolution between two
 extended behavior symbols.\par
Constructors: \haddocktt{res{\char 91}1-8{\char 93}{\char 91}1-4{\char 93}}.\par

\end{haddockdesc}
\begin{haddockdesc}
\item[\begin{tabular}{@{}l}
filter\ ::\ ExB\ b\ =>\ b\ Bool\ ->\ b\ a\ ->\ b\ a
\end{tabular}]\haddockbegindoc
Prefix name for the prefix operator \haddockid{/{\char '46}{\char '134}}.\par

\end{haddockdesc}
\begin{haddockdesc}
\item[\begin{tabular}{@{}l}
filter'\ ::\ ExB\ b\ =>\ Bool\ ->\ a\ ->\ b\ a
\end{tabular}]\haddockbegindoc
Same as \haddockid{filter} but takes base (non-extended) values as
 input arguments.\par

\end{haddockdesc}
\begin{haddockdesc}
\item[\begin{tabular}{@{}l}
degrade\ ::\ ExB\ b\ =>\ a\ ->\ b\ a\ ->\ a
\end{tabular}]\haddockbegindoc
Prefix name for the degrade operator \haddockid{/!{\char '134}}.\par

\end{haddockdesc}
\begin{haddockdesc}
\item[\begin{tabular}{@{}l}
ignore22
\end{tabular}]\haddockbegindoc
\haddockbeginargs
\haddockdecltt{::} & ExB b \\
                     \haddockdecltt{=>} & \haddockdecltt{(a1
                                                          -> a2
                                                             -> a1'
                                                                -> a2'
                                                                   -> (a1, a2))} & function of \haddocktt{Y\ +\ X} arguments \\
                                                                                   \haddockdecltt{->} & \haddockdecltt{a1} & \\
                                                                                                                             \haddockdecltt{->} & \haddockdecltt{a2} & \\
                                                                                                                                                                       \haddockdecltt{->} & \haddockdecltt{b a1'} & \\
                                                                                                                                                                                                                    \haddockdecltt{->} & \haddockdecltt{b a2'} & \\
                                                                                                                                                                                                                                                                 \haddockdecltt{->} & \haddockdecltt{(a1, a2)} & \\
\end{tabulary}\par
(image: fig/eqs-exb-pattern-ignore.png)\par
The \haddocktt{ignoreXY} pattern takes a function of \haddocktt{Y\ +\ X} arguments, \haddocktt{Y}
 basic inputs followed by \haddocktt{X} behavior-extended inputs. The \haddocktt{X}
 behavior-extended arguments are subjugated to a res, and the
 result is then degraded using the first \haddocktt{Y} arguments as
 fallback. The effect is similar to "ignoring" a the result of a
 res function if ∈ \emph{b}.\par
The main application of this pattern is as extended behavior
 wrapper for state machine functions which do not "understand"
 extended behavior semantics, i.e. it simply propagates the current
 state (∈ \emph{α}) if the inputs (their res) belongs
 to the set of extended values (∈ \emph{b}).\par
Constructors: \haddocktt{ignore{\char 91}1-4{\char 93}{\char 91}1-4{\char 93}}.\par

\end{haddockdesc}